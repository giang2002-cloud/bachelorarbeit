\chapter{INTRODUCTION}

\section{Background and Research Rationale}

\subsubsection{The Emergence of Social Sustainability Reporting (SSR)}
\begin{itemize}
    \item Under increasing stakeholder pressure, firms are progressively expected to disclose information related to their 
    corporate social responsibilities (CSR), particularly in the social domain \parencite{Monteiro2022}.
    \item Such disclosures are increasingly viewed as important for reinforcing stakeholder confidence and supporting the 
    firm's overall strategic positioning \parencite{Monteiro2022}.
    \item However, in practice, companies vary significantly in their readiness to implement CSR reporting.
    \item The social dimension has not received adequate attention \parencite{Heldal2024}.
    \item The social dimension is often viewed as vague, hard to quantify, and underemphasized, compared to environmental \parencite{Morais2018} and governance aspects.
\end{itemize}

\subsubsection{The Specific Role of the Software Industry}
\begin{itemize}
    \item Beyond sectors like retail, pharmaceuticals, and medical manufacturing, the social dimension of CSR reporting is gaining prominence in software services \parencite{Holder-Webb2009}.
    \item This is because software industry exerts substantial social impact through its indirect influence via digital platforms \parencite{Jimenez2024}.
    \item However, there is currently no clear or tailored SSR framework that reflects the industry's particularities, such as intangible assets, remote workforce, and agile organizational structures.
\end{itemize}

\section{Research Gap}
\begin{itemize}
    \item There is limited research on corporate readiness for social sustainability reporting during the transition to mandatory 
    regulations in the EU \parencite{Fiechter2022}.
    \item Most existing studies focus on ESG in general or primarily on environmental aspects \parencite{Mani2018}, with an emphasis 
    on industries like manufacturing, finance, or energy—sectors with more tangible datasets.
    \item Many current reports remain superficial and lack specific social indicators \parencite{Christensen2021, Fernandez2025}.
    \item There is an absence of sector-specific SSR frameworks \parencite{Bochkay2025}.
    \item It is still unclear which factors drive or hinder the readiness for SSR among software companies in Europe.
\end{itemize}

\section{Research Objectives}
\subsubsection{Aim}
To develop a structured assessment framework for measuring the readiness of software service companies in Europe to engage in social 
sustainability reporting.

\subsubsection{Objectives}
\begin{itemize}
    \item Identify key SSR criteria applicable to the software industry.
    \item Evaluate the SSR readiness of software companies using the developed criteria.
    \item Analyze the key drivers and barriers encountered in the implementation of social reporting.
    \item Provide practical recommendations to enhance SSR readiness for digital service enterprises.
\end{itemize}

\section{Scope and Limitations of Research}
\subsubsection{Scope}
\begin{itemize}
    \item The study focuses solely on the “Social” dimension of ESG; environmental and governance aspects are not analyzed in detail.
    \item The sample includes 30 medium- and large-sized software companies headquartered in Europe during 2022--2023.
    \item Data sources include annual reports, ESG reports and the SRN database.
\end{itemize}

\subsubsection{Limitations}
\begin{itemize}
    \item The analysis is restricted to the social pillar of ESG and does not address the environmental (E) or governance (G) components.
    \item The sample size is limited to 30 companies, which may not fully represent the entire software industry.
    \item No internal survey or interview data is used.
    \item Some evaluation criteria may involve a degree of subjectivity.
\end{itemize}

\section{Significance of the Study}
\subsubsection{Academic Contribution}
\begin{itemize}
    \item Addresses a research gap by proposing a sector-specific SSR framework.
    \item Offers a practical quantitative tool to assess social reporting capabilities.
\end{itemize}

\subsubsection{Practical Contribution}
\begin{itemize}
    \item Enables software companies to self-assess their readiness considering mandatory regulations such as the CSRD.
    \item Provides evidence-based insights for policymakers on the current state and support needs regarding SSR.
\end{itemize}

\section{Structure of the Research}
This thesis is organized as follows:
\begin{itemize}
    \item Chapter 1 introduces the background, research gap, objectives, scope, limitations, significance, and structure of the study.
    \item Chapter 2 reviews relevant literature, including definitions, sector context, theoretical frameworks, and prior research.
    \item Chapter 3 outlines the research methodology, including approach, data collection, assessment framework, and limitations.
    \item Chapter 4 presents findings and analysis from empirical research.
    \item Chapter 5 discusses the empirical results and their implications.
    \item Chapter 6 summarizes the research, key conclusions, contributions, recommendations, limitations, and suggestions for future research.
    \item References are provided at the end.
\end{itemize}