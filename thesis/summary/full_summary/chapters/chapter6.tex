\chapter{SUMMARY AND CONCLUSIONS}

\section{Research Summary and Conclusions}

\subsubsection{Research Summary (Proposal)}

The primary aim of this research was to develop and apply a Social Sustainability Reporting (SSR) 
readiness framework tailored to the European software sector. The research process involved:

\begin{itemize}
    \item Designing a comprehensive framework comprising 246 criteria based on established standards;
    \item Systematically scoring 30 software companies over the period 2022-2023;
    \item Analyzing SSR readiness patterns across thematic groups;
    \item Identifying key drivers and challenges influencing SSR implementation within the sector.
\end{itemize}

\subsubsection{Key Conclusions (Proposal)}

\begin{itemize}
    \item \textbf{Overall Readiness:} The analysis revealed that overall SSR readiness 
    among the sampled companies remains moderate, with substantial variation across organizations.

    \item \textbf{Strongest and Weakest Thematic Groups:} The strongest thematic categories were related to 
    workforce characteristics and health and safety, while collective bargaining 
    and social dialogue showed the weakest readiness.
    \item \textbf{Key Drivers:} Key enablers of SSR adoption included regulatory pressure (e.g., CSRD), 
    stakeholder expectations (e.g., from clients or investors), and reputational benefits tied to ESG positioning.
    
    \item \textbf{Major Challenges:} Primary barriers identified were resource constraints, 
    the absence of standardized social metrics, and a lack of tailored guidance for the software services sector.
\end{itemize}

\subsubsection{Unique Sector Traits}

\begin{itemize}
    \item The software sector exhibits distinct characteristics compared to other industries, including:
    \begin{itemize}
        \item Rapid innovation cycles;
        \item Intangible value creation (e.g., through intellectual capital);
        \item Heavy reliance on digital infrastructure.
    \end{itemize}
    \item These traits significantly influence how SSR practices are implemented and prioritized.
\end{itemize}

\section{Contributions of the Study}

\subsubsection{Academic Contribution}

\begin{itemize}
    \item \textbf{New Perspectives:}  
    Provides new insights into SSR readiness within digital service industries, 
    emphasizing sector-specific challenges and opportunities.
    
    \item \textbf{Theoretical Synthesis:}  
    Applies a multi-theoretical approach (Institutional Theory, Resource-Based View, and Stakeholder Theory) 
    to explain variations in SSR readiness across companies.
\end{itemize}

\subsubsection{Practical Contribution}

\begin{itemize}
    \item \textbf{Usable Framework:}  
    Introduces a simplified and practical SSR readiness framework that 
    enables software companies to assess their status and identify improvement areas.

\end{itemize}

\section{Recommendations}

\begin{itemize}
    \item \textbf{For Companies:}
    \begin{itemize}
        \item Treat SSR as a strategic capability;
        \item Invest in data infrastructure and assign dedicated personnel;
    \end{itemize}

    \item \textbf{For Policymakers and Standard-Setters:}
    \begin{itemize}
        \item Develop SSR standards tailored to the software/services sector;
        \item Provide support for companies through model reporting tools and open data platforms (e.g., APIs).
    \end{itemize}
\end{itemize}

\section{Research Limitations}

\begin{itemize}
    \item \textbf{Sample Size:}  
    The study is limited to 30 companies, all based in Europe.

    \item \textbf{Framework Testing:}  
    The SSR readiness framework is preliminary and has not yet been tested widely.

    \item \textbf{Data Sources:}  
    Only publicly available data were used; no interviews or internal data were collected.
\end{itemize}

\section{Suggestions for Future Research}

\begin{itemize}
    \item \textbf{Expand Sample and Coverage:}  
    Future studies should include more companies and extend to non-European regions.

    \item \textbf{Cross-Industry Application:}  
    Apply the framework to related service sectors to test generalizability.

    \item \textbf{Validation Methods:}  
    Incorporate surveys or expert interviews to validate the framework's design and scoring logic.

    \item \textbf{Statistical Analysis:}  
    Conduct advanced statistical analyses to explore correlations between SSR readiness and business or ESG performance metrics.
\end{itemize}
