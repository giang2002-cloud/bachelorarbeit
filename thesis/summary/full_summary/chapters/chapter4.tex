\chapter{FINDINGS AND ANALYSIS FROM EMPIRICAL RESEARCH}

\section{Overview of SSR Readiness Across the Sample}
\subsubsection{Sample Overview}
\begin{itemize}
    \item This section provides a brief description of the 30 software companies included in the sample, covering:
    \begin{itemize}
        \item Geographic distribution
        \item Company size (small, medium, large)
        \item Status of ESG/SSR report disclosure
    \end{itemize}
\end{itemize}

\begin{table}[H]
    \centering
    \caption{Basic Information of Sampled Companies (Appendix A)} 
    \begin{tabular}{l l l l}
        \textbf{Company} & \textbf{Country} & \textbf{Size} & \textbf{Status of ESG \& SSR report disclosure} \\    
    \end{tabular}
\end{table}

\begin{table}[H]
    \centering
    \caption{Sample Classification by Region and Company Size (Appendix B)} 
    
    \begin{tabular}{l l l}
        \textbf{Region/Size} & \textbf{Number of Companies} & \textbf{SSR Readiness Score} \\
    \end{tabular}
\end{table}

\subsubsection{Overall Results}
\begin{itemize}
    \item SSR readiness scores across the sample range (to be completed with data). 
    \item There is substantial variation in average scores across companies and countries.
\end{itemize}

\begin{table}[H]
    \centering
    \caption{SSR Readiness Score for Each Company}
    \begin{tabular}{l l l}
        \textbf{Company} & \textbf{Score accounding sub-sectors (0-246)} & \textbf{SSR Readiness Score (0-3)} \\
    \end{tabular}
\end{table}

\begin{figure}[H]
    \centering
    \caption{Histogram/Bar Chart of Readiness Score Distribution}
\end{figure}

\begin{table}[H]
    \centering
    \caption{Average Readiness Score by Country}
    \begin{tabular}{l l}
        \textbf{Country} & \textbf{Average Readiness Score} \\
        % Add data rows here
    \end{tabular}
\end{table}

\section{Readiness by Company Characteristics}
\subsubsection{Comparison of readiness based on:}
\begin{itemize}
    \item Company size (SMEs vs. large)
    \item Geographic location (Western vs. Central \& Eastern Europe)
    \item Business model
    \item ESG disclosure status (presence of standalone ESG report)
\end{itemize}

\begin{table}[H]
    \centering
    \caption{Readiness Score by Company Characteristics}
    \begin{tabular}{l l l l}
        \textbf{Attribute} & \textbf{Number of Companies} & \textbf{Average Readiness} & \textbf{Std. Dev.} \\
    \end{tabular}
\end{table}

\begin{figure}[H]
    \centering
    \caption{Bar Chart Comparing Readiness by Attribute Groups}
\end{figure}

\section{Analysis by the Framework Categories}
\subsubsection{Readiness across thematic groups: (To be altered by further analysis)}
\begin{itemize}
    \item Workforce Characteristics: Key themes across all categories
    \item Collective Bargaining and Social Dialogue: Often limited or absent in SMEs
    \item Compensation: Commonly disclosed and relatively easy to quantify
    \item Training and Development: Present but rarely detailed with clear metrics
    \item Health and Safety: Frequently addressed, though metrics vary
    \item Work-Life Balance: Mentioned, but coverage is often vague
    \item Human Rights: Unevenly addressed; more prevalent in larger firms
\end{itemize}

\begin{table}[H]
    \centering
    \caption{Readiness by Main Groups (Appendix C)}
\end{table}

\begin{figure}[H]
    \centering
    \caption{Bar Chart of 7 Main Group Readiness (Appendix D)}
\end{figure}

\subsubsection{Readiness by Mid-Groups}
\begin{itemize}
    \item Each main group is further divided into mid-groups, which provide more granular insights into specific reporting areas.
    \item Mid-groups reveal strengths and weaknesses within each thematic category, highlighting areas for improvement.
\end{itemize}

\begin{table}[H]
    \centering
    \caption{Readiness by Mid-Groups (Appendix E)}
\end{table}

\begin{figure}[H]
    \centering
    \caption{Bar Chart of Mid-Group Readiness (Appendix F)}
\end{figure}

\subsubsection{Top and Bottom Reporting Sub-Criteria}
\begin{itemize}
    \item Identify the top 5 sub-criteria with the highest reporting rates (e.g., >70\%).
    \item Identify the bottom 5 sub-criteria with the lowest reporting rates (e.g., <30\%).
    \item Discuss potential reasons for these patterns, such as ease of measurement, stakeholder demand, or sector norms.
\end{itemize}

\begin{table}[H]
    \centering
    \caption{Top 5 Highest and Lowest Reporting Sub-Criteria}
\end{table}

\section{Empirical Drivers and Challenges}
\subsubsection{Key Drivers of Reporting Readiness}
\begin{itemize}
    \item Sub-criteria and mid-groups with high disclosure rates (e.g., >70\%) 
    indicate areas where reporting is either well established, easy to quantify, or commonly requested by stakeholders. 
    \item These drivers include elements such as compensation transparency, health and safety incidents, and DEI statements. 
    \item Their structure, familiarity, or measurability contribute to higher readiness.
\end{itemize}

\subsubsection{Key Challenges Hindering Reporting Readiness}
\begin{itemize}
    \item Low-scoring criteria (e.g.,<30\% disclosure rate) signal inherent challenges.       
    \item These include vague or qualitative dimensions (e.g., collective bargaining quality, freedom of expression), 
    lack of existing frameworks (e.g., social dialogue outcomes), or sensitivity of topics (e.g., grievances, wage gaps). 
     \item These challenges hinder readiness not because companies refuse to report, but due to structural complexity, 
     ambiguity, or lack of metrics.
\end{itemize}

\begin{table}[H]
    \centering
    \caption{Summary of Key Drivers and Barriers}
    \begin{tabular}{l l l}
        \textbf{Category} & \textbf{Description} & \textbf{Number of Companies} \\
        % Add data rows here
    \end{tabular}
\end{table}

\section{Summary of Findings}
\begin{itemize}
    \item Strong and weak reporting categories
    \item Factors positively/negatively affecting SSR readiness
    \item Prepares for discussion
\end{itemize}