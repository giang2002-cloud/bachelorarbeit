\chapter{LITERATURE REVIEW}

\section{Definitions and Sector Context}

\subsection{Overview of Social Sustainability Reporting}

\subsubsection{Definition and Scope}
\begin{itemize}
    \item Social sustainability reporting in this research highlights the S-Pillar in CSR. 
    \item In the literature, terms such as CSR, 'extended external reporting', 'ESG reporting', and 
    'sustainability reporting' are often used interchangeably \parencite{Edge2022, Fiechter2022}. 
    \item Sustainability reporting encompasses disclosures on social, environmental, and governance dimensions, 
    with the social aspect focusing on issues such as labor practices, diversity, and community engagement.
\end{itemize}

\subsubsection{Roles and Value of SSR}
\begin{itemize}
    \item SSR enhances transparency and builds trust with stakeholders (investors, employees, customers, governments). 
    \item It serves as a key component within broader ESG strategies and CSR agendas.
\end{itemize}

\subsubsection{Relevant Conceptual and Standards Frameworks}
\begin{itemize}
    \item Corporate Sustainability Reporting Directive (CSRD):
        \begin{itemize}
            \item The EU's CSRD, effective from January 2024, is a mandatory regulation requiring large and 
            listed companies to disclose ESG-related information. 
            \item  It establishes a legal basis for sustainability reporting and 
            reinforces the importance of the social dimension across sectors.       
        \end{itemize}

    \item European Sustainability Reporting Standards (ESRS):
        \begin{itemize}
            \item Developed under the CSRD, the ESRS provide detailed reporting requirements. 
            \item In particular, ESRS S1 focuses on social sustainability related to an organization's own workforce, 
        \end{itemize}

    \item GRI 401--405: Voluntary global standards covering employment, labor relations, diversity, and equal pay.
    \item ISO 26000: Voluntary international guideline promoting social responsibility across stakeholder groups.
\end{itemize}

\subsection{Overview of Reporting Readiness}
\subsubsection{Definition}
\begin{itemize}
    \item Reporting readiness refers to the extent to which an organization possesses the capacity, systems, data, 
    and commitment required to conduct high-quality, reliable SSR. 
    \item It represents a transitional stage between awareness of SSR and successful implementation.
\end{itemize}

\subsubsection{Key Characteristics}
\begin{itemize}
    \item \textbf{Disclosure Scope}: Coverage of key social topics such as working conditions, diversity, and employee well-being.
    \item \textbf{Stakeholder Engagement}: Involvement of internal actors like HR teams and management in the reporting process.
    \item \textbf{Data Availability}: Access to measurable and reliable social data that enables transparent reporting.
    \item \textbf{Maturity Levels}: Readiness can be assessed using multi-level scales reflecting progression from basic to advanced reporting practices.
\end{itemize}

\subsubsection{Link to Organizational Capabilities}
\begin{itemize}
    \item Reporting readiness is viewed as an organizational capability that can evolve over time.
    \item It can be measured through structured frameworks proposed in the literature or developed in specific studies.
\end{itemize}

\subsection{Overview of Software Services Sector}
\subsubsection{Sector Characteristics}
\begin{itemize}
    \item The software services sector is a service-based industry highly reliant on human capital and intangible assets. 
    \item It is rapidly evolving with remote and hybrid work environments, facing high competition for talent and increasing expectations for attractive workplace conditions.
\end{itemize}

\subsubsection{Role of Social Aspects}
\begin{itemize}
    \item Employees are core assets; thus, working conditions, benefits, and training are of strategic importance. 
    \item ESG performance increasingly influences the ability to attract B2B clients and investors.
\end{itemize}

\subsubsection{Challenges for SSR}
\begin{itemize}
    \item Lack of standardized quantitative data and measurement systems for social indicators \parencite{Gibbons2024}.
    \item Existing standards provide limited industry-specific SSR guidance for digital service firms.
    \item Many companies are SMEs with limited resources for non-financial reporting.
\end{itemize}

\section{Theoretical Framework}
\subsection{Institutional Theory (DiMaggio \& Powell, 1983)}

\subsubsection{Coercive Pressure}
\begin{itemize}
    \item Legal and regulatory requirements such as the CSRD, ESRS, and EU directives mandate ESG disclosure. 
    \item Transparency laws, human rights policies, and DEI reporting obligations from public institutions and industry coalitions.
\end{itemize}

\subsubsection{Normative Pressure}
\begin{itemize}
    \item Expectations from professional associations, major clients, and investors to comply with recognized reporting norms. 
\end{itemize}

\subsubsection{Mimetic Pressure}
\begin{itemize}
    \item Firms imitate industry leaders or competitors when SSR is seen to deliver reputational or HR-related advantages. 
    \item This is particularly salient in the software sector, where best practices are quickly observed and replicated due to intense competition.
\end{itemize}

\subsection{Resource-Based View (RBV) (Barney, 1991)}
\subsubsection{Readiness as an Outcome of Internal Resources}
\begin{itemize}
    \item Firms with ESG data systems, HR software, and dedicated sustainability personnel are better equipped to measure and report 
    social outcomes.
    \item These internal elements are viewed as intangible assets that support the development of SSR capability.
\end{itemize}

\subsubsection{SSR Readiness as a Strategic Capability}
\begin{itemize}
    \item Readiness is considered a difficult-to-imitate capability embedded in organizational culture and structure.
    \item It enhances brand reputation, attracts top talent, and improves access to ESG-oriented capital.
\end{itemize}

\subsection{Stakeholder Theory (Freeman, 1984)}
\subsubsection{Key Stakeholders Influencing SSR Readiness}
\begin{itemize}
    \item Investors: Increasingly attentive to social risks and ESG transparency.
    \item Employees: Expect fair, diverse, and growth-oriented workplaces.
    \item Customers: Prefer socially responsible partners, particularly in tech supply chains.
    \item Communities and Local Authorities: Demand ethical behavior, equality, and positive social contributions.
\end{itemize}

\subsubsection{Benefits of SSR from a Stakeholder Perspective}
\begin{itemize}
    \item SSR builds trust and credibility with key stakeholder groups.
    \item Supports a responsible corporate image—particularly important in sectors like software, where competition for talent is intense.
\end{itemize}

\section{Literature Review}

\subsection{Overview of Prior Research}
\subsubsection{Sector-Specific SSR Studies}
\begin{itemize}
    \item Many studies on CSR and sustainable development have focused on industry-specific contexts such as manufacturing, construction, 
    and mining \parencite{Ye2020, Afshari2022}. 
    \item CSR reporting is more prevalent in environmentally sensitive industries \parencite{Lock2016}. 
    \item High-tech sectors, particularly software, have received comparatively limited attention. 
    \item Earlier research acknowledges that the factors influencing sustainability reporting practices are often mixed and 
    context-dependent \parencite{Fifka2013}, 
    \item There remains no consistent framework to explain what drives or hinders SSR readiness.
\end{itemize}

\subsubsection{ESG in the Technology Sector}
\begin{itemize}
    \item The “Social” component is often the weakest in ESG disclosures due to lack of standardized indicators, perceived immateriality, 
    or symbolic reporting \parencite{Christensen2021, Reitmaier2024}.
\end{itemize}

\subsection{Existing Readiness Assessment Models}
\subsubsection{Prior Frameworks}
\begin{itemize}
    \item \textbf{\textcite{Nava2023}:} Emphasize the need for robust SDG measurement and reporting frameworks.
    \item \textbf{\textcite{ElBaz2022}:} Propose a three-stage sustainability readiness framework (Approach - Deployment - Results) 
    for Industry 4.0 adoption.
    \item \textbf{\textcite{Afshari2022}:} Present a multi-dimensional classification of Social Sustainability Indicators (SSIs), but 
    not a formal readiness model.
    \item \textbf{\textcite{Zopounidis2020}:} Develop a multi-criteria ESG disclosure assessment framework, enabling benchmarking but 
    not tailored to SSR readiness or the software sector.
    \item \textbf{\textcite{Siew2015}:} Reviews corporate sustainability reporting tools, which mainly assess current reporting performance.
    \item \textbf{\textcite{Okongwu2013}:} Propose maturity models for sustainability reporting, focusing on governance, system integration, 
    and leadership commitment.
    \item \textbf{\textcite{Barletta2021}:} Organisational Sustainability Readiness (OSR) Model assesses internal capability to implement 
    sustainability strategies, adaptable to software services.
\end{itemize}

\subsubsection{Identified Gap}
\begin{itemize}
    \item Absence of SSR readiness models that reflect the specific characteristics of the software industry.
    \item Need for an integrated framework drawing from ESRS, ISO, GRI standards, and the context of the digital economy.
\end{itemize}

\subsection{Drivers and Challenges in Prior Research}
\subsubsection{Drivers}
\begin{itemize}
    \item Regulatory compliance pressures—such as GRI, CSRD, or SASB standards—encourage ESG disclosure in the software sector \parencite{Reitmaier2024, Bochkay2025}.
    \item Investors and corporate clients increasingly expect transparency and alignment with ESG principles \parencite{Bonnefon2025, Dai2021}.
    \item Enhancing corporate reputation is a key driver for improved social sustainability reporting \parencite{Reitmaier2024}.
    \item Firms recognize strategic and operational benefits from participating in collaborative CSR initiatives \parencite{Dai2021}.
    \item Strong societal and ethical practices are positively linked to higher market valuation \parencite{Chouaibi2021}.
\end{itemize}

\subsubsection{Challenges}
\begin{itemize}
    \item Social reporting standards remain insufficiently defined in many industry-specific contexts, including software and IT services \parencite{Bochkay2025}.
    \item Many firms face limitations in data infrastructure and ESG reporting systems, hindering consistent and reliable disclosure \parencite{Troshani2024, ElBaz2022, Jona2023, Belal2015}.
    \item Stakeholder relationships, particularly in B2B contexts, are often characterized by power asymmetries that reduce external reporting pressure \parencite{Dai2021}.
    \item Reputational concerns can lead to superficial or symbolic social reporting practices, increasing the risk of greenwashing \parencite{Reitmaier2024, Belal2015}.
    \item High short-term costs and expensive assurance processes discourage investment in comprehensive social reporting, despite long-term benefits \parencite{Dai2021, Najjar2023}.
\end{itemize}

\subsection{Analytical Orientation Informed by Prior Research}
\begin{itemize}
    \item Previous studies have highlighted several factors influencing SSR readiness, including 
        \begin{itemize}
        \item regulatory pressure (e.g., CSRD, ESRS)
        \item stakeholder expectations (e.g., from investors and clients)
        \item internal systems and organizational capabilities \parencite{Christensen2021}
        \item resource constraints
        \end{itemize}
    \item Foundational theories such as Institutional Theory, Resource-Based View (RBV), and Stakeholder Theory provide a basis for understanding the key drivers and barriers to social reporting.
    \item Studies on sustainability reporting structure and indicator design provided 
    the conceptual basis for organizing related indicators into thematically coherent mid-groups.
        \begin{itemize}
        \item Based on this structure, the study aims to construct an assessment framework 
        for evaluating SSR readiness in the software services industry.
        \item The scoring results are then analyzed to uncover prevailing patterns, key drivers, 
        and common challenges in current reporting practices.
        \end{itemize}
\end{itemize}