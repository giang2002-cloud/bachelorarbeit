\chapter{KẾT QUẢ VÀ PHÂN TÍCH TỪ NGHIÊN CỨU THỰC NGHIỆM}

\section{Mức độ sẵn sàng SSR theo đặc điểm tổ chức}
Các đặc điểm của doanh nghiệp được xem xét để trả lời Câu hỏi nghiên cứu 1, tập trung vào việc mô tả các mô hình quan sát được trong mức độ sẵn sàng SSR.
Những đặc điểm này bao gồm cả các thuộc tính sau này được phân tích như là yếu tố thúc đẩy hoặc rào cản đối với mức độ sẵn sàng báo cáo, cũng như các đặc điểm khác không cho thấy sức mạnh giải thích và chỉ mang tính mô tả.

\section{Mức độ sẵn sàng theo các chiều khung phân tích}

\subsubsection{1. Sẵn sàng theo nhóm chính}
Điểm số tổng thể về mức độ sẵn sàng được phân tích trên bảy nhóm chủ đề chính trong khung đánh giá.

\subsubsection{2. Sẵn sàng theo nhóm trung gian}
\begin{itemize}
\item Mỗi nhóm chính được chia thành các nhóm trung gian, cung cấp cái nhìn chi tiết hơn về từng lĩnh vực báo cáo cụ thể.
\item Các nhóm trung gian chỉ ra điểm mạnh và điểm yếu trong từng hạng mục, làm nổi bật các lĩnh vực cần cải thiện.
\end{itemize}

\subsubsection{3. Các tiêu chí phụ cao và thấp nhất}
Một phân tích tập trung vào năm tiêu chí phụ được báo cáo nhiều nhất và ít nhất minh họa nơi mà các công ty tự tin nhất trong công bố, cũng như những thách thức lớn vẫn còn tồn tại.

\section{Các yếu tố thúc đẩy và thách thức thực nghiệm của mức độ sẵn sàng SSR}
Để phân biệt giữa các đặc điểm tổ chức mang tính mô tả và những yếu tố có tính giải thích, nghiên cứu này dựa trên ba lăng kính lý thuyết: Lý thuyết thể chế, Quan điểm dựa trên nguồn lực (RBV), và Lý thuyết các bên liên quan.

Không phải mọi đặc điểm của công ty quan sát được trong mẫu đều đủ điều kiện làm yếu tố giải thích.
Ví dụ, tuổi đời hay vị trí công ty, dù mang ý nghĩa mô tả, nhưng không cho thấy mô hình nhất quán và thiếu cơ sở lý thuyết, do đó không được xem là yếu tố thúc đẩy hoặc rào cản.

\subsection{Các yếu tố thúc đẩy chính}

\begin{table}[H]
\centering
\caption{Các yếu tố thúc đẩy báo cáo SSR và diễn giải lý thuyết}
\begin{tabular}{p{6cm}|p{8cm}}
\subsubsection{Yếu tố thúc đẩy} & \subsubsection{Diễn giải lý thuyết} \
\hline
Áp lực quy định từ CSRD, EFRAG, SEC; sự chấp nhận ESG sớm hoặc nghiêm ngặt tại quốc gia trụ sở chính của công ty
& Lý thuyết thể chế: Áp lực cưỡng chế thúc đẩy thay đổi hành vi \
\hline
Yêu cầu từ khách hàng và các nhà đầu tư lớn, thể hiện trong mô hình B2C hoặc B2B
& Lý thuyết các bên liên quan: Kỳ vọng từ các bên liên quan khuyến khích minh bạch và nâng cao chất lượng báo cáo \
\hline
Thiếu hụt nhân tài $\rightarrow$ SSR được sử dụng cho xây dựng thương hiệu tuyển dụng, đặc biệt trong các công ty nhấn mạnh DEI, đào tạo, hoặc cân bằng công việc – cuộc sống
& Lý thuyết các bên liên quan & RBV: SSR trở thành lợi thế cạnh tranh trong tuyển dụng và danh tiếng \
\end{tabular}
\end{table}

\subsection{Các thách thức chính}

\begin{table}[H]
\centering
\caption{Các thách thức đặc thù ngành và diễn giải lý thuyết}
\begin{tabular}{p{6cm}|p{8cm}}
\subsubsection{Thách thức} & \subsubsection{Diễn giải lý thuyết} \
\hline
Khó khăn trong việc thu thập và chuẩn hóa dữ liệu phi tài chính & RBV: Phản ánh sự thiếu hụt hệ thống, công cụ và nhân sự – năng lực tổ chức chưa phát triển \
\hline
Lao động phi chính thức, toàn cầu hóa, làm việc từ xa & Lý thuyết thể chế: Bối cảnh phân mảnh làm suy yếu sự thống nhất pháp lý và áp lực cưỡng chế \
\hline
Thiếu các tiêu chuẩn xã hội đặc thù ngành & Lý thuyết thể chế: Áp lực chuẩn mực còn yếu; chưa hình thành “chuẩn mực xã hội” cho ngành \
\hline
SMEs thiếu ngân sách/nhân sự ESG & RBV: SMEs thường thiếu nguồn lực chiến lược để xây dựng năng lực báo cáo nội bộ \
\end{tabular}
\end{table}

