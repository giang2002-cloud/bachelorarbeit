\chapter{FINDINGS AND ANALYSIS FROM EMPIRICAL RESEARCH}

\section{SSR Readiness by Organizational Characteristics}
Company characteristics are examined to address Research Question 1, which focuses on describing observable patterns 
in SSR readiness. These characteristics include both attributes that are later explored as potential 
drivers or barriers to reporting readiness, as well as other traits that do not exhibit explanatory power 
and remain solely descriptive in nature.

\section{Thematic Readiness by Framework Dimensions}
\subsubsection{1. Readiness by Main Groups}
The overall readiness scores were analyzed across the seven main thematic groups in the assessment framework.

\subsubsection{2. Readiness by Mid-Groups}
\begin{itemize}
    \item Each main group is further divided into mid-groups, which provide more granular insights into specific reporting areas.
    \item Mid-groups reveal strengths and weaknesses within each thematic category, highlighting areas for improvement.
\end{itemize}

\subsubsection{3. Top and Bottom Sub-Criteria}
A focused analysis of the five most and least frequently reported sub-criteria illustrates 
where companies are most confident in their disclosures and where significant challenges persist. 

\section{Empirical Drivers and Challenges of SSR Readiness}
To distinguish between mere organizational characteristics and explanatory factors, this study 
draws on three theoretical lenses: Institutional Theory, the Resource-Based View, and Stakeholder Theory.

Not all company characteristics observed in the sample qualify as explanatory factors.
For example, company age or location, while descriptively relevant, show no consistent patterns 
and lack theoretical grounding, thus are not considered drivers or barriers.

\subsection{Key Drivers}
\begin{table}[H]
    \centering
    \caption{Drivers of SSR Reporting and Theoretical Interpretation}
    \begin{tabular}{p{6cm}|p{8cm}}
        \subsubsection{Driver} & \subsubsection{Theoretical Interpretation} \\
        \hline
        Regulatory pressure from CSRD, EFRAG, SEC; early or strict ESG adoption in firm's country of headquarters
         & Institutional Theory: Coercive pressure driving behavioral change \\
        \hline
        Demands from customers and large investors, reflected in B2C or B2B model 
        & Stakeholder Theory: Stakeholder expectations incentivize transparency and reporting quality \\
        \hline
        Talent shortage $\rightarrow$ SSR used for employer branding, especially among companies 
        highlighting DEI, training, or work-life balance 
        & Stakeholder \& RBV: SSR becomes a competitive advantage in recruitment and reputation \\
    \end{tabular}
\end{table}

\subsection{Key Challenges}

\begin{table}[H]
    \centering
    \caption{Industry-Specific Challenges and Theoretical Interpretation}
    \begin{tabular}{p{6cm}|p{8cm}}
        \subsubsection{Challenge} & \subsubsection{Theoretical Interpretation} \\
        \hline
        Difficulty in collecting and standardizing non-financial data & RBV: Reflects lack of systems, tools, and personnel—organizational capabilities not yet developed \\
        \hline
        Informal labor, globalization, remote work & Institutional Theory: Fragmented settings weaken legal coherence and coercive pressure \\
        \hline
        Lack of sector-specific social standards & Institutional Theory: Normative pressure is underdeveloped; no established "social norms" for the sector \\
        \hline
        SMEs lack ESG budget/personnel & RBV: SMEs often lack the strategic resources to build internal reporting capabilities
    \end{tabular}
\end{table}


