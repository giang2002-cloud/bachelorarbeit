\chapter{SUMMARY AND CONCLUSIONS}

\section{Research Summary}
The primary aim of this research was to develop and apply a Social and Sustainability Reporting (SSR) readiness framework tailored for the European software sector. The research process involved designing a comprehensive framework comprising 246 criteria, systematically scoring 30 software companies during 2022--2023, and analyzing their SSR readiness levels. The study further explored key drivers and challenges influencing SSR adoption in the sector.

\section{Key Conclusions}
\subsubsection{Overall Readiness}
The analysis revealed that the overall SSR readiness among the sampled companies remains moderate, with significant variation across organizations.

\subsubsection{Strongest and Weakest Thematic Groups}
Companies generally performed best in governance and data security criteria, while environmental and social impact reporting were identified as the weakest areas.

\subsubsection{Common Drivers}
Key drivers promoting SSR included regulatory pressure, stakeholder expectations, and reputational benefits.

\subsubsection{Major Challenges}
Major challenges hindering comprehensive SSR reporting were resource constraints, lack of standardized metrics, and limited sector-specific guidance.

\subsubsection{Unique Sector Traits}
Compared to other industries, the software sector exhibited unique traits such as rapid innovation cycles, intangible value creation, and a reliance on digital infrastructure, all of which shape SSR practices.

\section{Contributions of the Study}
\subsection{Academic Contribution}
\subsubsection{New Perspectives}
This study adds new perspectives on SSR readiness within digital service industries, highlighting sector-specific challenges and opportunities.

\subsubsection{Theoretical Synthesis}
By applying a synthesis of Institutional Theory, Resource-Based View (RBV), and Stakeholder Theory, the research explains variations in SSR readiness across companies.

\subsection{Practical Contribution}
\subsubsection{Usable Framework}
The study offers a simplified and practical SSR readiness framework for software companies, enabling organizations to assess their current position and identify improvement areas.

\subsubsection{Support for SMEs}
Small and medium-sized enterprises (SMEs) can use the framework to benchmark their SSR journey and pursue collaborative or simplified solutions.

\section{Recommendations}
\subsubsection{For Companies}
Companies should treat SSR as a strategic capability, prioritize investment in data infrastructure and dedicated personnel, and SMEs are encouraged to seek collaborative solutions or adopt simplified SSR tools.

\subsubsection{For Policymakers and Standard-Setters}
Policymakers and standard-setters should develop SSR standards tailored to the software/services sector and support SMEs through model tools and open data platforms, such as APIs.

\section{Research Limitations}
\subsubsection{Sample Size}
The study is limited by a small sample size of 30 companies, all based in Europe.

\subsubsection{Framework Testing}
The SSR readiness framework remains preliminary and has not yet been widely tested.

\subsubsection{Data Sources}
The research relies solely on publicly available data, with no interviews conducted.

\section{Suggestions for Future Research}
\subsubsection{Expand Sample and Coverage}
Future studies should expand the sample size and include companies from other regions.

\subsubsection{Cross-Industry Application}
Testing the framework in other service industries, such as IT and finance, would enhance its generalizability.

\subsubsection{Validation Methods}
Incorporating surveys or expert interviews can help validate the framework's applicability.

\subsubsection{Statistical Analysis}
Deeper statistical analysis is recommended to examine the relationship between SSR readiness and business or ESG performance metrics.