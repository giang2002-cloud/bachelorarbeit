\chapter{TÓM TẮT VÀ KẾT LUẬN}

\subsubsection{Tóm tắt nghiên cứu và kết luận}
Nghiên cứu này đã xây dựng và áp dụng một khung đánh giá mức độ sẵn sàng SSR cho 30 công ty phần mềm.

\subsubsection{Đóng góp của nghiên cứu}
\begin{itemize}
\item Nghiên cứu đưa ra các kết luận then chốt liên quan đến mức độ sẵn sàng tổng thể, điểm mạnh và điểm yếu theo chủ đề, cũng như các yếu tố thúc đẩy và rào cản chính.
\item Nghiên cứu xác định các mô hình đáng chú ý trong mức độ sẵn sàng SSR, nhấn mạnh ảnh hưởng của áp lực pháp lý, kỳ vọng từ các bên liên quan, và năng lực tổ chức.
\end{itemize}

\subsubsection{Khuyến nghị}
Nghiên cứu đưa ra các khuyến nghị thực tiễn dành cho doanh nghiệp, nhà hoạch định chính sách và tổ chức ban hành chuẩn mực.

\subsubsection{Hạn chế phương pháp, đạo đức nghiên cứu và hướng phát triển tương lai}

\subsubsection{Hạn chế}
Nghiên cứu này bị giới hạn bởi quy mô mẫu gồm 30 công ty, phụ thuộc vào dữ liệu công khai, và sử dụng một khung SSR sơ bộ chưa được kiểm chứng độc lập.

\subsubsection{Đạo đức nghiên cứu}
Nghiên cứu tuân thủ các chuẩn mực đạo đức khi chỉ sử dụng dữ liệu công khai, đảm bảo không công ty nào bị chỉ trích hoặc nêu đích danh.

\subsubsection{Hướng nghiên cứu tương lai}
Nghiên cứu tương lai nên mở rộng mẫu theo phạm vi địa lý, áp dụng khung nghiên cứu sang các ngành khác, và kiểm chứng thông qua phỏng vấn hoặc nghiên cứu tình huống.