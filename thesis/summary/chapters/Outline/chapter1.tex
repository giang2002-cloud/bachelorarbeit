\chapter{GIỚI THIỆU}

\subsubsection{Đoạn 1: Căn cứ của Nghiên cứu}

Khi các mối quan tâm về xã hội và tính bền vững ngày càng nổi bật trong chương trình nghị sự của doanh nghiệp, 
các thực tiễn báo cáo đang phát triển trên các ngành công nghiệp. Các công ty trong lĩnh vực dịch vụ phần mềm ngày càng được kỳ vọng 
tiết lộ hiệu suất bền vững xã hội của họ, 
được thúc đẩy bởi các áp lực thể chế, các cân nhắc chiến lược nội bộ, 
và kỳ vọng của các bên liên quan.

\subsubsection{Đoạn 2: Khoảng Trống Nghiên cứu}

Mặc dù tầm quan trọng ngày càng tăng của SSR, vẫn thiếu các khung khổ toàn diện 
được thiết kế riêng cho những thách thức độc đáo mà lĩnh vực dịch vụ phần mềm phải đối mặt. 
Tài liệu hiện có thường bỏ qua bối cảnh và nhu cầu cụ thể của ngành công nghiệp này, 
dẫn đến một khoảng trống mà nghiên cứu này nhằm mục đích giải quyết.

\subsubsection{Đoạn 3: Mục tiêu Nghiên cứu}

Nghiên cứu này nhằm phát triển một khung đánh giá có cấu trúc để đo lường 
mức độ sẵn sàng của các công ty dịch vụ phần mềm tham gia 
vào báo cáo bền vững xã hội. Nó đánh giá các thực tiễn tiết lộ thông tin, 
xác định các thách thức, và đề xuất các chiến lược cải thiện.

\subsubsection{Đoạn 4: Phạm vi và Hạn chế của Nghiên cứu}
Nghiên cứu chỉ tập trung vào khía cạnh xã hội của ESG, phân tích 30 
công ty phần mềm sử dụng dữ liệu có sẵn công khai. 
Nó không bao gồm các khía cạnh E/G hoặc thu thập dữ liệu sơ cấp.

\subsubsection{Đoạn 5: Ý nghĩa của Nghiên cứu}

Về mặt học thuật, luận án lấp đầy khoảng trống nghiên cứu thông qua 
việc phát triển một khung khổ cụ thể cho lĩnh vực đánh giá báo cáo bền vững xã hội
trong dịch vụ phần mềm. Về mặt thực tiễn, nó cung cấp một công cụ định lượng 
cho các công ty đánh giá mức độ sẵn sàng SSR của họ trong bối cảnh các yêu cầu quy định 
như CSRD.

\subsubsection{Đoạn 6: Câu hỏi Nghiên cứu (RQs)}
Nghiên cứu này nhằm trả lời các câu hỏi sau:
\begin{itemize}
    \item RQ1: Mức độ sẵn sàng báo cáo bền vững xã hội hiện tại 
    trong lĩnh vực dịch vụ phần mềm là gì?
    \item RQ2: Những yếu tố tổ chức nào đóng vai trò là động lực chính hoặc rào cản 
    ảnh hưởng đến mức độ sẵn sàng của các công ty cho báo cáo bền vững xã hội?
\end{itemize}

\subsubsection{Đoạn 7: Cấu trúc của Nghiên cứu}
Luận án này được tổ chức như sau:
\begin{itemize}
    \item Chương 2 xem xét tài liệu liên quan, bao gồm các định nghĩa, 
    bối cảnh ngành, khung lý thuyết, và nghiên cứu trước đây.
    \item Chương 3 nêu phương pháp nghiên cứu, bao gồm cách tiếp cận, 
    thu thập dữ liệu và khung đánh giá.
    \item Chương 4 trình bày các phát hiện và phân tích từ nghiên cứu thực nghiệm.
    \item Chương 5 thảo luận về ý nghĩa thực tiễn và học thuật của các phát hiện.
    \item Chương 6 tóm tắt nghiên cứu, kết luận chính, đóng góp, 
    khuyến nghị, hạn chế, đạo đức nghiên cứu và đề xuất cho nghiên cứu tương lai.
\end{itemize}