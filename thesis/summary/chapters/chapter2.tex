\chapter{LITERATURE REVIEW}

\section{Definitions and Sector Context}

\subsection{Overview of Social Sustainability Reporting}
Social sustainability reporting in this research highlights the S-Pillar in CSR. In the literature, terms such as CSR, ‘extended external reporting’, ‘ESG reporting’, and ‘sustainability reporting’ are often used interchangeably \parencite{Edge2022, Fiechter2022}. Sustainability reporting encompasses disclosures on social, environmental, and governance dimensions, with the social aspect focusing on issues such as labor practices, diversity, and community engagement.

\subsubsection{Roles and Value of SSR}
SSR enhances transparency and builds trust with stakeholders (investors, employees, customers, governments). It serves as a key component within broader ESG strategies and CSR agendas.

\subsubsection{Relevant Conceptual and Standards Frameworks}
\begin{itemize}
    \item GRI Standards (Global Reporting Initiative), particularly GRI 401–405.
    \item ESRS S1 (European Sustainability Reporting Standards – Social 1: Own Workforce).
    \item ISO 26000 on social responsibility.
    \item SSR orientation as defined in the EU’s CSRD directive.
\end{itemize}

\subsection{Overview of Reporting Readiness}
Reporting readiness refers to the extent to which an organization possesses the capacity, systems, data, and commitment required to conduct high-quality, reliable SSR. It represents a transitional stage between awareness of SSR and successful implementation.

\subsubsection{Key Characteristics}
\begin{itemize}
    \item \textbf{Governance:} Clear policies and assigned responsibilities for SSR.
    \item \textbf{Data \& Infrastructure:} Availability of appropriate data and measurement systems (e.g., HRIS, ESG tools).
    \item \textbf{Engagement:} Involvement of internal stakeholders (HR teams, leadership) and awareness of SSR benefits.
    \item \textbf{Maturity Scale:} Readiness can be assessed using a maturity scale (e.g., 0–1 or 0–3) for each criterion group.
\end{itemize}

Readiness is considered an organizational capability that can evolve over time and can be measured through structured frameworks—such as the one developed in this study.

\subsection{Overview of Software Services Sector}
The software services sector is a service-based industry highly reliant on human capital and intangible assets. It is rapidly evolving with remote and hybrid work environments, facing high competition for talent and increasing expectations for attractive workplace conditions.

\subsubsection{Role of Social Aspects}
Employees are core assets; thus, working conditions, benefits, and training are of strategic importance. ESG performance increasingly influences the ability to attract B2B clients and investors.

\subsubsection{Challenges for SSR}
\begin{itemize}
    \item Lack of standardized quantitative data and measurement systems for social indicators \parencite{Gibbons2024}.
    \item SSR is often perceived as unrelated to core business outcomes.
    \item Existing standards provide limited industry-specific SSR guidance for digital service firms.
    \item Many companies are SMEs with limited resources for non-financial reporting.
\end{itemize}

\section{Theoretical Framework}

\subsection{Institutional Theory (DiMaggio \& Powell, 1983)}
\subsubsection{Coercive Pressure}
Legal and regulatory requirements such as the CSRD, ESRS, and EU directives mandate ESG disclosure. Transparency laws, human rights policies, and DEI reporting obligations from public institutions and industry coalitions.

\subsubsection{Normative Pressure}
Expectations from professional associations, major clients, and investors to comply with recognized reporting norms. Standards like GRI and ISO 26000 increasingly function as “soft law.”

\subsubsection{Mimetic Pressure}
Firms imitate industry leaders or competitors when SSR is seen to deliver reputational or HR-related advantages. This is particularly salient in the software sector, where best practices are quickly observed and replicated due to intense competition.

\textit{Relevance to the Study:} Explains variations in SSR readiness as a result of differing levels and types of institutional pressure across firms.

\subsection{Resource-Based View (RBV) (Barney, 1991)}
\subsubsection{Readiness as an Outcome of Internal Resources}
Firms with established ESG data systems, HR software, and dedicated sustainability personnel are better positioned to measure and report social outcomes. These elements are viewed as intangible assets forming the basis for SSR capability.

\subsubsection{SSR Readiness as a Strategic Capability}
A difficult-to-imitate capability embedded in organizational culture and structure. Contributes to brand reputation, ability to attract top talent, and access to ESG-oriented capital.

\textit{Relevance to the Study:} Explains why firms in the same industry and of similar size may display differing levels of SSR readiness—depending on their internal capabilities and strategic investment in social reporting.

\subsection{Stakeholder Theory (Freeman, 1984)}
\subsubsection{Key Stakeholders Influencing SSR Readiness}
\begin{itemize}
    \item Investors: Increasingly attentive to social risks and ESG transparency.
    \item Employees: Expect fair, diverse, and growth-oriented workplaces.
    \item Customers: Prefer socially responsible partners, particularly in tech supply chains.
    \item Communities and Local Authorities: Demand ethical behavior, equality, and positive social contributions.
\end{itemize}

\subsubsection{Benefits of SSR from a Stakeholder Perspective}
Builds trust and credibility with key stakeholder groups. Supports a responsible corporate image—particularly important in sectors like software, where competition for talent is intense.

\textit{Relevance to the Study:} Helps explain how varying levels of stakeholder pressure lead to differences in firms' proactive engagement and readiness for SSR.

\section{Literature Review}

\subsection{Overview of Prior Research}
Many studies on CSR and sustainable development have focused on industry-specific contexts such as manufacturing, construction, and mining \parencite{Ye2020, Afshari2022}. CSR reporting is more prevalent in environmentally sensitive industries \parencite{Lock2016}. High-tech sectors, particularly software, have received comparatively limited attention. Earlier research acknowledges that the factors influencing sustainability reporting practices are often mixed and context-dependent \parencite{Fifka2012}, and there remains no consistent framework to explain what drives or hinders SSR readiness.

\subsubsection{ESG in the Technology Sector}
The “Social” component is often the weakest in ESG disclosures due to lack of standardized indicators, perceived immateriality, or symbolic reporting \parencite{Christensen2021, Reitmaier2025}.

\subsection{Existing Readiness Assessment Models}
\begin{itemize}
    \item \textbf{Nava et al. (2023):} Emphasize the need for robust SDG measurement and reporting frameworks.
    \item \textbf{El Baz et al. (2022):} Propose a three-stage sustainability readiness framework (Approach – Deployment – Results) for Industry 4.0 adoption.
    \item \textbf{Afshari et al. (2022):} Present a multi-dimensional classification of Social Sustainability Indicators (SSIs), but not a formal readiness model.
    \item \textbf{Zopounidis et al. (2020):} Develop a multi-criteria ESG disclosure assessment framework, enabling benchmarking but not tailored to SSR readiness or the software sector.
    \item \textbf{Siew (2015):} Reviews corporate sustainability reporting tools, which mainly assess current reporting performance.
    \item \textbf{Okongwu et al. (2022):} Propose maturity models for sustainability reporting, focusing on governance, system integration, and leadership commitment.
    \item \textbf{Barletta et al. (2021):} Organisational Sustainability Readiness (OSR) Model assesses internal capability to implement sustainability strategies, adaptable to software services.
\end{itemize}

\subsubsection{Identified Gap}
\begin{itemize}
    \item Absence of SSR readiness models that reflect the specific characteristics of the software industry.
    \item Need for an integrated framework drawing from ESRS, ISO, GRI standards, and the context of the digital economy.
\end{itemize}

\subsection{Drivers and Challenges of SSR Implementation – Insights from Prior Studies}
\subsubsection{Drivers}
\begin{itemize}
    \item Regulatory compliance pressures (e.g., GRI, CSRD) \parencite{Reitmaier2025}, voluntary industry-specific ESG disclosure standards (e.g., SASB for Software \& IT Services) \parencite{Bochkay2025}.
    \item Expectations from stakeholders like investors regarding transparency and ESG alignment \parencite{Bonnefon2022}, and corporate customers across supply chains \parencite{Dai2019}.
    \item Corporate reputation \parencite{Reitmaier2025}.
    \item Operational and strategic benefits from collaborative CSR initiatives \parencite{Dai2019}.
    \item Positive relationship between societal and ethical practices and market valuation \parencite{Chouaibi2021}.
\end{itemize}

\subsubsection{Challenges}
\begin{itemize}
    \item Insufficiently defined social reporting standards in industry-specific contexts \parencite{Bochkay2025}.
    \item Limited data infrastructure and reporting systems \parencite{Troshani2024, ElBaz2022, Jona2023, Belal2016}.
    \item Power asymmetry in stakeholder relationships, especially in B2B contexts \parencite{Dai2019}.
    \item Risk of superficial reporting driven by reputational concerns (greenwashing) \parencite{Reitmaier2025, Belal2016}.
    \item Short-term cost concerns overshadow long-term benefits \parencite{Dai2019}, high cost for auditing \parencite{Najjar2023}.
\end{itemize}

\subsection{Analytical Orientation Informed by Prior Research}
Previous studies have highlighted several factors influencing SSR readiness, including regulatory pressure (e.g., CSRD, ESRS), stakeholder expectations (e.g., from investors and clients), internal systems and organizational capabilities \parencite{Christensen2021}, and resource constraints.

Foundational theories such as Institutional Theory, Resource-Based View (RBV), and Stakeholder Theory provide a basis for understanding the key drivers and barriers to social reporting.

This study adopts an exploratory approach with the following aims:
\begin{itemize}
    \item To construct an assessment framework for SSR readiness.
    \item To analyze scoring results and derive insights into prevailing trends, drivers, and challenges in the software industry.
\end{itemize}