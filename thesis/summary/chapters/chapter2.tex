\chapter{CƠ SỞ LÝ THUYẾT, NGHIÊN CỨU TRƯỚC ĐÂY VÀ NỀN TẢNG KHÁI NIỆM}

\section{Các khung lý thuyết cho mức độ sẵn sàng báo cáo bền vững xã hội (SSR) trong lĩnh vực dịch vụ phần mềm}

\subsubsection{1. Đoạn mở đầu (khoảng 1/3 trang): Giới thiệu vai trò của lý thuyết trong nghiên cứu}

Các khung lý thuyết cung cấp nền tảng để hiểu sự khác biệt về mức độ sẵn sàng báo cáo bền vững xã hội (SSR) giữa các doanh nghiệp dịch vụ phần mềm.

\subsubsection{2. Nội dung chính (khoảng 2 trang): Tổng quan các lý thuyết chủ chốt}

\vspace{8pt}
\subsubsection{a. Lý thuyết thể chế (Institutional Theory)}

Lý thuyết thể chế \parencite{DiMaggio1983} giải thích cách các yếu tố bên ngoài như quy định pháp lý (ví dụ: CSRD, ESRS), chuẩn mực ngành và hành vi đồng đẳng tạo ra áp lực ảnh hưởng đến động lực báo cáo bền vững xã hội của các doanh nghiệp phần mềm.

\subsubsection{b. Quan điểm dựa trên nguồn lực (Resource-Based View – RBV)}

Quan điểm dựa trên nguồn lực \parencite{Barney1991} nhấn mạnh rằng mức độ sẵn sàng SSR phụ thuộc vào năng lực nội tại của doanh nghiệp, và việc thiếu hụt các nguồn lực này có thể hạn chế khả năng sẵn sàng ngay cả khi chịu áp lực mạnh mẽ từ bên ngoài.

\subsubsection{c. Lý thuyết các bên liên quan (Stakeholder Theory)}

Lý thuyết các bên liên quan \parencite{Freeman1984} coi SSR là phản ứng mang tính chiến lược nhằm đáp ứng kỳ vọng của các bên liên quan then chốt.

\subsubsection{3. Đoạn kết (khoảng 1/3 trang): Sự tích hợp và tính liên quan với nghiên cứu}

Ba lý thuyết này bổ sung cho nhau, cung cấp những giải thích vừa riêng biệt vừa gắn kết về mức độ sẵn sàng SSR, và sẽ được sử dụng như công cụ phân tích để diễn giải kết quả thực nghiệm ở Chương 5.

\section{Nền tảng và các nghiên cứu trước về Báo cáo Bền vững Xã hội}

\subsubsection{Đoạn 1: Tổng quan và nghiên cứu trước về SSR}

Báo cáo bền vững xã hội trong nghiên cứu này tập trung vào trụ cột xã hội (S-Pillar) trong báo cáo CSR.
Tổng quan về SSR đã được trình bày \parencite{Edge2022,Fiechter2022}, và các chuẩn mực liên quan \parencite{Dechow2023} sẽ được sử dụng ở Chương 3 (CSRD, ESRS, GRI).
Nghiên cứu trước đây về SSR trong lĩnh vực phần mềm còn khan hiếm \parencite{Ye2020,Afshari2022}, và các công bố xã hội thường còn yếu \parencite{Christensen2021,Reitmaier2024}.

\subsubsection{Đoạn 2: Các chỉ số về lực lượng lao động nội bộ theo ESRS S1: Định nghĩa và nghiên cứu trước}

Theo ESRS S1, “own workforce” bao gồm tất cả nhân viên có quan hệ hợp đồng trực tiếp với doanh nghiệp.
Các yêu cầu công bố trải rộng trên nhiều lĩnh vực chủ đề.

Trong nghiên cứu này, chúng được hợp nhất thành bảy nhóm chính để phân tích có hệ thống: đặc điểm lực lượng lao động, thương lượng tập thể và đối thoại xã hội, đãi ngộ, đào tạo, sức khỏe và an toàn, cân bằng công việc – cuộc sống, và nhân quyền.

Nghiên cứu trước: Các khía cạnh then chốt như đào tạo và phát triển, cân bằng công việc – cuộc sống, tôn trọng và hòa nhập, thù lao và phúc lợi, cũng như sự tham gia của người lao động thường chỉ được đề cập hạn chế \parencite{Greig2021}.
Dù nhiều doanh nghiệp lớn đã công bố dữ liệu đa dạng giới, nhưng các công bố này thường giới hạn và gắn với khoảng cách lương giới tính lớn hơn, cho thấy động cơ mang tính danh tiếng \parencite{Huang2022}.

\section{Đặc điểm ngành và thực hành báo cáo trong lĩnh vực dịch vụ phần mềm}

\subsubsection{Đoạn 1: Đặc điểm ngành}

Ngành dịch vụ phần mềm là một ngành dựa trên dịch vụ, phụ thuộc cao vào vốn nhân lực và tài sản vô hình.

\subsubsection{Đoạn 2: Thực hành báo cáo trong dịch vụ phần mềm}

Thực hành báo cáo trong lĩnh vực dịch vụ phần mềm đang dần phát triển, các doanh nghiệp ngày càng nhận thức rõ tầm quan trọng của bền vững xã hội. Tuy nhiên, nhiều doanh nghiệp vẫn gặp khó khăn trong triển khai và đo lường hiệu quả.

\section{Khái niệm hóa và các cách tiếp cận trước đây về mức độ sẵn sàng báo cáo}

\subsubsection{Đoạn 1: Định nghĩa mức độ sẵn sàng báo cáo}

Mức độ sẵn sàng báo cáo được hiểu là mức độ mà một tổ chức sở hữu năng lực, hệ thống, dữ liệu và cam kết cần thiết để thực hiện báo cáo SSR chất lượng cao và đáng tin cậy.
Khái niệm này bao gồm cả sự tồn tại của các yếu tố nền tảng và tính trọng yếu (xác định và ưu tiên các vấn đề xã hội có liên quan đến ngành).

\subsubsection{Đoạn 2: Các cách tiếp cận trước đây về mức độ sẵn sàng báo cáo}

\subsubsection{1. Các khung sẵn sàng trong những lĩnh vực khác:}
Nghiên cứu trước trong các lĩnh vực như Công nghiệp 4.0 \parencite{ElBaz2022}, đổi mới xanh \parencite{Zhang2020}, và sản xuất bền vững \parencite{Barletta2021} đã áp dụng rộng rãi khung đo lường sẵn sàng, từ mô hình theo giai đoạn, mô hình trưởng thành đến sẵn sàng, đến các đánh giá dựa trên yếu tố thúc đẩy và rào cản \parencite{Govindan2023}. Tuy nhiên, các cách tiếp cận này chưa được áp dụng cho SSR.

\subsubsection{2. Sẵn sàng/chuẩn bị trong báo cáo CSR:}
Nghiên cứu trước về báo cáo CSR chủ yếu đánh giá mức độ sẵn sàng thông qua các chuẩn tham chiếu như ESRS và phân tích thể chế, thay vì đo lường trực tiếp mức độ sẵn sàng \parencite{Filho2025,Shabana2017}.

\subsubsection{3. Các cách tiếp cận đo lường:}
Các cách tiếp cận đo lường trong báo cáo CSR chủ yếu dựa trên phương pháp chấm điểm nhằm đánh giá tuân thủ và hiệu quả, hơn là đánh giá mức độ sẵn sàng \parencite{Tobias2022,Papoutsi2020,Gai2023,Nicolo2025}.

\subsubsection{4. Đo lường mức độ sẵn sàng CSR bằng phương pháp chấm điểm:}
Nghiên cứu về mức độ sẵn sàng CSR còn hạn chế, chủ yếu tập trung ở cấp quốc gia, chú trọng ESG nói chung và phần lớn sử dụng phương pháp chấm điểm \parencite{Nguyen2024,Montero2025}.

\subsubsection{5. Cơ sở cho cách tiếp cận của nghiên cứu này:}
Vì SSR theo ESRS/GRI được tiêu chuẩn hóa cao và giàu tiêu chí, nên phương pháp chấm điểm vừa khả thi vừa cho phép đo lường mức độ sẵn sàng một cách minh bạch, có thể tái lập.

\section{Các yếu tố thúc đẩy và rào cản dựa trên nghiên cứu trước}

\subsubsection{Đoạn 1: Các yếu tố thúc đẩy báo cáo bền vững xã hội}

Các yếu tố thúc đẩy SSR bao gồm:
\begin{itemize}
\item tuân thủ quy định pháp lý \parencite{Reitmaier2024,Bochkay2025}
\item kỳ vọng ESG từ nhà đầu tư và khách hàng \parencite{Bonnefon2025,Dai2021}
\item nâng cao danh tiếng \parencite{Reitmaier2024}
\item lợi ích từ hợp tác CSR \parencite{Dai2021}
\item thực hành đạo đức và giá trị thị trường \parencite{Chouaibi2021}
\end{itemize}

\subsubsection{Đoạn 2: Các rào cản đối với báo cáo bền vững xã hội}

Các rào cản của SSR bao gồm:
\begin{itemize}
\item thiếu các tiêu chuẩn đặc thù ngành \parencite{Bochkay2025}
\item hạn chế về hạ tầng dữ liệu và hệ thống ESG \parencite{Troshani2024,ElBaz2022,Jona2023,Belal2015}
\item bất cân xứng quyền lực giữa các bên liên quan trong B2B \parencite{Dai2021}
\item nguy cơ báo cáo hình thức và “greenwashing” \parencite{Reitmaier2024,Belal2015}
\item chi phí cao và tốn kém cho kiểm toán đảm bảo \parencite{Dai2021,Najjar2023}
\end{itemize}