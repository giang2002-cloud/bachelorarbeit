\chapter{THEORETICAL BACKGROUND, PRIOR LITERATURE, AND CONCEPTUAL FOUNDATIONS}

\section{Theoretical frameworks for social sustainability reporting (SSR) readiness in software services}

\subsubsection{1. Opening Paragraph (ca. 1/3 page): Introduction to the role of theory in this research}

Theoretical frameworks provide a foundation for understanding the variation 
in social sustainability reporting (SSR) readiness across software service firms.

\subsubsection{2. Main body paragraphs (ca. 2 page): Overview of key theories}

\vspace{8pt}
    \subsubsection{a. Institutional Theory}

Institutional Theory (DiMaggio \& Powell, 1983) explains how external forces such as regulatory mandates (e.g., CSRD, ESRS),
industry norms, and peer behavior exert pressures that influence the motivation of software firms in 
social sustainability reporting.

    \subsubsection{b. Resource-Based View (RBV)}

The Resource-Based View (Barney, 1991) highlights that SSR readiness depends on a firm's internal capabilities 
and that a lack of such resources may limit readiness even under strong external pressure.

    \subsubsection{c. Stakeholder Theory}

Stakeholder Theory (Freeman, 1984) views SSR as a strategic response to meet the expectations of key stakeholders.

\subsubsection{3. Closing paragraph (ca. 1/3 page): Integration and relevance to the study}

These theories complement each other by offering distinct yet interconnected explanations of SSR readiness, 
which will serve as analytical tools to interpret empirical results in Chapter 5.

\section{Foundations and Prior Literature on Social Sustainability Reporting}
\subsubsection{Paragraph 1: Overview of Social Sustainability Reporting}
Social sustainability reporting in this research highlights the S-Pillar in CSR Reporting.
Overview of SSR is presented (Edge 2022; Fiechter, Hitz, and
Lehmann 2022), and relevant standards to be used in Chapter 3 (CSRD, ESRS, GRI) are defined.

\subsubsection{Paragraph 2: Prior Literature on Social Sustainability Reporting}
Prior research on SSR in software is scarce (Ye et al. 2020; Afshari et al. 2022) and 
social disclosures are often weak (Christensen et al. 2021; Reitmaier et al. 2024).


\section{Sectoral Characteristics and Reporting Practices in Software Services}
\subsubsection{Paragraph 1: Sector Characteristics}
The software services sector is a service-based industry highly reliant on human capital and intangible assets.

\subsubsection{Paragraph 2: Reporting Practices in the Software Services}
Reporting practices in the software services sector are evolving, with firms increasingly recognizing 
the importance of social sustainability. However, many still struggle with effective implementation and measurement.

\section{Conceptualizations and Prior Approaches to Reporting Readiness}
\subsubsection{Paragraph 1: Definition of Reporting Readiness}
Reporting readiness refers to the extent to which an organization possesses the capacity, systems, data,
and commitment required to conduct high-quality, reliable SSR.

\subsubsection{Paragraph 2: Prior Approaches to Reporting Readiness}

\subsubsection{1. Readiness frameworks in other sectors:}
Prior research in sectors such as Industry 4.0 (El Baz et al., 2022), green innovation (Zhang, 2020),
and sustainable manufacturing (Barletta et al., 2021) has widely applied readiness frameworks,
ranging from stage-based models to maturity-to-readiness adaptations, 
and in some cases structuring assessments around drivers and barriers (Govindan, 2023).
However, these have not yet been used for SSR.

\subsubsection{2. Readiness/preparedness in CSR reporting:}
Prior research on CSR reporting has assessed readiness mainly through benchmarks 
such as ESRS and institutionalisation analyses rather than measuring readiness 
(Filho et al., 2025) , (Shabana, 2026).

\subsubsection{3. Measurement approaches:}
Measurement approaches in CSR reporting have primarily relied on scoring methods
aimed at compliance and performance rather than assessing readiness
(Tobias, 2022), (Sardianou, 2021), (Papoutsi \& Sodhi, 2020), (Gai, 2023), (Nicolo, 2025).

\subsubsection{4. CSR reporting readiness measurement with score-based approaches:}
Research on CSR reporting readiness is still limited, confined largely to the national level, 
and primarily employs scoring-based methods
(Nguyen, 2023), (Montero, 2025).

\subsubsection{5. Rationale for this study's approach:}
Given that SSR under ESRS/GRI is highly standardised and criteria-rich, 
a scoring-based approach is both feasible and enables transparent, reproducible readiness measurement.

\section{Literature-Based Drivers and Barriers}
\subsubsection{Paragraph 1: Drivers of Social Sustainability Reporting}
Drivers of social sustainability reporting include 
\begin{itemize}
    \item regulatory compliance, (Reitmaier, Schultze, and Vollmer 2024; Bochkay,
Hales, and Serafeim 2025)
    \item investor and client ESG expectations, (Bonnefon et al. 2025; Dai, Liang, and Ng 2021)
    \item reputation enhancement, (Reitmaier, Schultze, and Vollmer 2024)
    \item benefits from CSR collaboration, (Dai, Liang, and Ng 2021)
    \item ethical practices and market valuation, (S. Chouaibi and J. Chouaibi 2021).
\end{itemize}

\subsubsection{Paragraph 2: Barriers to Social Sustainability Reporting}
Barriers to social sustainability reporting encompass 
\begin{itemize}
    \item undefined industry-specific standards (Bochkay et al. 2025)
    \item limited data infrastructure and ESG systems (Troshani \& Rowbottom 2024; El Baz et al. 2022; Jona \& Soderstrom 2023; Belal \& Owen 2015)
    \item stakeholder power asymmetries in B2B (Dai et al. 2021)
    \item symbolic reporting and greenwashing risk (Reitmaier et al. 2024; Belal \& Owen 2015)
    \item high costs and assurance expenses (Dai et al. 2021; Najjar \& Yasin 2023)
\end{itemize}