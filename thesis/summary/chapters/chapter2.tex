\chapter{THEORETICAL BACKGROUND, PRIOR LITERATURE, AND CONCEPTUAL FOUNDATIONS}

\section{Theoretical frameworks for social sustainability reporting (SSR) readiness in software services}

\subsubsection{1. Opening Paragraph (ca. 1/3 page): Introduction to the role of theory in this research}

Theoretical frameworks provide a foundation for understanding the variation 
in social sustainability reporting (SSR) readiness across software service firms.

\subsubsection{2. Main body paragraphs (ca. 2 page): Overview of key theories}

\vspace{8pt}
    \subsubsection{a. Institutional Theory}

Institutional Theory (DiMaggio \& Powell, 1983) explains how external forces such as regulatory mandates (e.g., CSRD, ESRS),
industry norms, and peer behavior exert pressures that influence the motivation of software firms in 
social sustainability reporting.

    \subsubsection{b. Resource-Based View (RBV)}

The Resource-Based View (Barney, 1991) highlights that SSR readiness depends on a firm's internal capabilities 
and that a lack of such resources may limit readiness even under strong external pressure.

    \subsubsection{c. Stakeholder Theory}

Stakeholder Theory (Freeman, 1984) views SSR as a strategic response to meet the expectations of key stakeholders.

\subsubsection{3. Closing paragraph (ca. 1/3 page): Integration and relevance to the study}

These theories complement each other by offering distinct yet interconnected explanations of SSR readiness, 
which will serve as analytical tools to interpret empirical results in Chapter 5.

\section{Foundations and Prior Literature on Social Sustainability Reporting}
\subsubsection{Paragraph 1: Overview of Social Sustainability Reporting}
Social sustainability reporting in this research highlights the S-Pillar in CSR Reporting.

\subsubsection{Paragraph 2: Prior Literature on Social Sustainability Reporting}



\section{Sectoral Characteristics and Reporting Practices in Software Services}
\subsubsection{Paragraph 1: Sector Characteristics}
The software services sector is a service-based industry highly reliant on human capital and intangible assets.

\subsubsection{Paragraph 2: Reporting Practices in the Software Services}
Reporting practices in the software services sector are evolving, with firms increasingly recognizing 
the importance of social sustainability. However, many still struggle with effective implementation and measurement.

\section{Conceptualizations and Prior Approaches to Reporting Readiness}
\subsubsection{Paragraph 1: Definition of Reporting Readiness}
Reporting readiness refers to the extent to which an organization possesses the capacity, systems, data,
and commitment required to conduct high-quality, reliable SSR.

\subsubsection{Paragraph 2: Existing Frameworks}
Several frameworks have been proposed to assess reporting readiness, 
but they often lack specificity for the software services sector. 
Existing models may not fully capture the unique challenges and opportunities faced by firms in this industry.

\section{Literature-Based Drivers and Barriers}
\subsubsection{Paragraph 1: Drivers of Social Sustainability Reporting}
Drivers of social sustainability reporting include regulatory compliance, stakeholder expectations, 
and the desire to enhance brand reputation.

\subsubsection{Paragraph 2: Barriers to Social Sustainability Reporting}
Barriers to social sustainability reporting encompass a lack of standardized metrics, 
insufficient data collection processes, and limited awareness of social sustainability issues among key stakeholders.
