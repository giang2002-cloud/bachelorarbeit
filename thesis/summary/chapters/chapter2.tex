\chapter{LITERATURE REVIEW}

\section{Definitions and Sector Context}

\subsection{Overview of Social Sustainability Reporting}

\subsubsection{Definition and Scope}
\begin{itemize}
    \item Social sustainability reporting in this research highlights the 
    S-Pillar in CSR Reporting. 
\end{itemize}

\subsubsection{Roles and Value of SSR}
\begin{itemize}
    \item SSR enhances transparency and builds trust with stakeholders (investors, employees, customers, governments). 
    \item It serves as a key component within broader ESG strategies and CSR agendas.
\end{itemize}

\subsubsection{Relevant Conceptual and Standards Frameworks}
\begin{itemize}
    \item Corporate Sustainability Reporting Directive (CSRD)
    \item European Sustainability Reporting Standards (ESRS)
    \item GRI 401--405
    \item ISO 26000
\end{itemize}

\subsection{Overview of Reporting Readiness}
\subsubsection{Definition}
\begin{itemize}
    \item Reporting readiness refers to the extent to which an organization possesses the capacity, systems, data, 
    and commitment required to conduct high-quality, reliable SSR. 
\end{itemize}

\subsubsection{Link to Organizational Capabilities}
\begin{itemize}
    \item Reporting readiness is viewed as an organizational capability that can evolve over time.
    \item It can be measured through structured frameworks proposed in the literature or developed in specific studies.
\end{itemize}

\subsection{Overview of Software Services Sector}
\subsubsection{Sector Characteristics}
\begin{itemize}
    \item The software services sector is 
    a service-based industry highly reliant on human capital and intangible assets. 
\end{itemize}

\subsubsection{Role of Social Aspects}
\begin{itemize}
    \item Employees are core assets; thus, working conditions, benefits, and training are of strategic importance. 
    \item ESG performance increasingly influences the ability to attract B2B clients and investors.
\end{itemize}

\subsubsection{Challenges for SSR}
\begin{itemize}
    \item Lack of standardized quantitative data and measurement systems for social indicators.
    \item Existing standards provide limited industry-specific SSR guidance for digital service firms.
\end{itemize}

\section{Theoretical Framework}
\subsection{Institutional Theory (DiMaggio \& Powell, 1983)}
\begin{itemize}
    \item Coercive pressures: laws (CSRD, ESRS), DEI policies.
    \item Normative pressures: expectations from clients/investors.
    \item Mimetic pressures: copying leaders to gain reputational or HR advantages.
\end{itemize}

\subsection{Resource-Based View (RBV) (Barney, 1991)}
\begin{itemize}
    \item Internal assets like HR systems, ESG teams, and data infrastructure drive readiness.
    \item SSR capability is strategic, hard to replicate, and linked to performance.
\end{itemize}

\subsection{Stakeholder Theory (Freeman, 1984)}
\begin{itemize}
    \item Key actors: investors, employees, clients, and local communities.
    \item SSR improves trust and strengthens competitive advantage in talent markets.
\end{itemize}

\section{Overview of Prior Research}

\subsection{Existing Literature on Social Reporting in Software Sector}
\subsubsection{Sector-Specific SSR Studies}
\begin{itemize}
    \item Focus has been on manufacturing, construction, and energy.
    \item Limited attention to software services despite their growing ESG relevance.
    \item No unified model yet explains SSR readiness across sectors.
\end{itemize}

\subsubsection{ESG in the Technology Sector}
\begin{itemize}
    \item The “S” dimension is often underreported or symbolic.
    \item Existing ESG reporting lacks specificity for digital service firms.
\end{itemize}

\subsection{Existing Readiness Assessment Frameworks}
\subsubsection{Prior Frameworks}
\begin{itemize}
    \item Multiple models exist (e.g., Siew, El Baz, Barletta), but most:
    \begin{itemize}
        \item Are generic or industry-neutral.
        \item Focus on reporting output rather than readiness.
        \item Do not capture the digital/service-specific context.
    \end{itemize}
\end{itemize}

\subsubsection{Identified Gap}
\begin{itemize}
    \item Lack of software-specific SSR readiness frameworks.
    \item Need to integrate ESRS, GRI, ISO, and digital economy traits.
\end{itemize}

\subsection{Literature-Based Drivers and Challenges}
\subsubsection{Drivers}
\begin{itemize}
    \item Regulatory compliance (CSRD, GRI, SASB).
    \item Stakeholder expectations (investors, clients).
    \item Brand reputation and CSR partnerships.
    \item Social performance linked to market valuation.
\end{itemize}

\subsubsection{Challenges}
\begin{itemize}
    \item Lack of sector-specific standards.
    \item Weak ESG infrastructure and data systems.
    \item Symbolic reporting due to reputational risk.
    \item High short-term costs and limited capacity in SMEs.
\end{itemize}

\subsection{Analytical Orientation Informed by Prior Research}
\begin{itemize}
    \item This thesis builds on:
        \begin{itemize}
        \item Regulatory and stakeholder drivers.
        \item Organizational capacity and digital sector constraints.
        \item Theoretical grounding in Institutional Theory, RBV, and Stakeholder Theory.
        \end{itemize}
\end{itemize}