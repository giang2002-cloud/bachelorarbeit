\chapter{INTRODUCTION}

\section{Background and Research Rationale}

\subsubsection{The Emergence of Social Sustainability Reporting (SSR)}
\begin{itemize}
    \item Social sustainability reporting (SSR) 
    is gaining importance under increasing regulatory and stakeholder pressure.
    \item However, most research still emphasizes environmental or governance aspects. 
    \item The software sector, despite its rising social impact, lacks tailored SSR frameworks and shows varying degrees of readiness.
\end{itemize}

\section{Research Gap}
\begin{itemize}
    \item There is limited research on corporate readiness for social 
    sustainability reporting during the transition to mandatory 
    regulations in the EU.
    \item Existing SSR models are mostly generic and overlook sector-specific challenges.
\end{itemize}

\section{Research Objectives}
\subsubsection{Aim}
To develop and apply a framework 
for assessing SSR readiness in the European software sector.

\subsubsection{Objectives}
\begin{itemize}
    \item Identify key SSR criteria applicable to the software industry.
    \item Evaluate the SSR readiness of software companies using the developed criteria.
    \item Analyze the key drivers and barriers encountered in the implementation of social reporting.
    \item Provide practical recommendations to enhance SSR readiness for digital service enterprises.
\end{itemize}

\section{Scope and Limitations of Research}
\subsubsection{Scope}
\begin{itemize}
    \item Focuses on the “Social” pillar in ESG for 
    30 European software firms (2022-2023), based on public reports.
\end{itemize}

\subsubsection{Limitations}
\begin{itemize}
    \item Excludes E/G aspects, relies on secondary data, 
    no surveys/interviews, and includes some subjectivity in scoring.
\end{itemize}

\section{Significance of the Study}
\subsubsection{Academic Contribution}
\begin{itemize}
    \item Addresses a research gap by proposing a sector-specific SSR framework.
    \item Offers a practical quantitative tool to assess social reporting capabilities.
\end{itemize}

\subsubsection{Practical Contribution}
\begin{itemize}
    \item Enables software companies to self-assess their readiness considering mandatory regulations such as the CSRD.
    \item Provides evidence-based insights for policymakers on the current state and support needs regarding SSR.
\end{itemize}

\section{Structure of the Research}
This thesis is organized as follows:
\begin{itemize}
    \item Chapter 1 introduces the background, research gap, objectives, scope, limitations, significance, and structure of the study.
    \item Chapter 2 reviews relevant literature, including definitions, sector context, theoretical frameworks, and prior research.
    \item Chapter 3 outlines the research methodology, including approach, data collection, assessment framework, and limitations.
    \item Chapter 4 presents findings and analysis from empirical research.
    \item Chapter 5 discusses the empirical results and their implications.
    \item Chapter 6 summarizes the research, key conclusions, contributions, recommendations, limitations, and suggestions for future research.
    \item References are provided at the end.
\end{itemize}