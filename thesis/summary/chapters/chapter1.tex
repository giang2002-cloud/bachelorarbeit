\chapter{INTRODUCTION}

\subsubsection{Paragraph 1: The Rationale for the Research}

As social and sustainability concerns gain prominence in corporate agendas, 
reporting practices are evolving across industries. Firms in the software services sector are increasingly expected to 
disclose their social sustainability performance, 
driven by institutional pressures, internal strategic considerations, 
and stakeholder expectations.

\subsubsection{Paragraph 2: Research Gap}

Despite the growing importance of SSR, there is a lack of comprehensive frameworks 
tailored to the unique challenges faced by the software services sector. 
Existing literature often overlooks the specific context and needs of this 
industry, resulting in a gap that this research aims to address.

\subsubsection{Paragraph 3: Research Objectives}

This study aims to develop a structured assessment framework to measure 
the readiness of software service companies in Europe to engage 
in social sustainability reporting. It evaluates disclosure practices, 
identifies challenges, and proposes improvement strategies.

\subsubsection{Paragraph 4: Scope and Limitations of Research}
The study focuses only on the social dimension of ESG, analyzing 30 European 
software firms using publicly available data. 
It does not cover E/G dimensions or collect primary data.

\subsubsection{Paragraph 5: Significance of the Study}

Academically, the thesis fills a research gap through the 
development of a sector-specific framework for assessing social sustainability
reporting in software services. Practically, it offers a quantitative tool 
for companies to evaluate their SSR readiness in light of regulatory 
requirements such as the CSRD.

\subsubsection{Paragraph 6: Research Questions (RQs)}
This research aims to answer the following questions:
\begin{itemize}
    \item RQ1: What is the current level of social sustainability reporting readiness 
    in the European software services sector?
    \item RQ2: Which organizational factors act as key drivers or barriers 
    influencing firms' readiness for social sustainability reporting?
\end{itemize}

\subsubsection{Paragraph 7: Structure of the Research}
This thesis is organized as follows:
\begin{itemize}
    \item Chapter 2 reviews relevant literature, including definitions, 
    sector context, theoretical frameworks, and prior research.
    \item Chapter 3 outlines the research methodology, including approach, 
    data collection and assessment framework.
    \item Chapter 4 presents findings and analysis from empirical research.
    \item Chapter 5 discusses the practical and academic implications of the findings.
    \item Chapter 6 summarizes the research, key conclusions, contributions, 
    recommendations, limitations, research ethics and suggestions for future research.
\end{itemize}