\chapter{RESEARCH METHODOLOGY}

\section{Research Approach and Design}
This study adopts an exploratory approach, aimed at developing and applying an assessment framework for SSR readiness.

\subsubsection{Methodology}
A mixed-methods design was employed, combining:
\begin{itemize}
    \item Qualitative content analysis of company reports;
    \item Quantitative descriptive statistics;
    \item Framework design aligned with Design Science Research (DSR) principles.
\end{itemize}

Three main phases of the research:
\begin{enumerate}
    \item Developing the framework, consisting of 7 main groups and 246 sub-criteria;
    \item Collecting data from 30 European software companies (2022--2023);
    \item Descriptive analysis to assess levels of readiness and identify key patterns.
\end{enumerate}

\subsubsection{Methodological Objectives}
\begin{itemize}
    \item Assess the social reporting capabilities of software companies;
    \item Identify major drivers and barriers in practice;
    \item Propose directions for improving the framework and aligning SSR capabilities with emerging EU regulations (CSRD/ESRS).
\end{itemize}

\section{Research Subjects and Data Collection}
\subsection{Sampling Criteria}
\subsubsection{Sample Size}
30 software companies listed or disclosing ESG information publicly in Europe (during 2022--2023).

\subsubsection{Inclusion Criteria}
\begin{itemize}
    \item Companies operating in the software services sector, headquartered or primarily active in Europe;
    \item Accessibility of financial reports, ESG/sustainability reports, or other public documents for 2022 and 2023 through SRN.
\end{itemize}

\subsection{Data Collection}
\subsubsection{Primary Sources}
\begin{itemize}
    \item Annual reports, ESG/sustainability reports, company websites;
    \item SRN -- a centralized sustainability reporting platform.
\end{itemize}

\subsubsection{Data Set}
\begin{itemize}
    \item Compiled SSR-related information corresponding to 246 framework criteria for each of the 30 software companies in 2022 and 2023.
    \item Data coded and recorded using a structured scoring sheet (Excel or qualitative coding software).
\end{itemize}

\section{Assessment Framework and Data Analysis}
\subsection{Developing the SSR Readiness Framework}
\subsubsection{Reference Standards}
\begin{itemize}
    \item ESRS S1 -- Own Workforce;
    \item GRI 2 -- General Disclosures, GRI 401--406;
    \item Supplementary inputs from ISO and industry-specific documents.
\end{itemize}

\subsubsection{Framework Structure}
\begin{itemize}
    \item A total of 246 sub-criteria (reporting items), partially derived from ESRS/GRI and supplemented by practical or sectoral inputs.
    \item 7 main indicator groups, each containing 2--3 mid-groups, reflecting specific social dimensions such as:
    \begin{itemize}
        \item Workforce Characteristics
        \item Working Conditions
        \item Social Benefits
        \item Training and Development
        \item Diversity and Inclusion
        \item Labor Relations
        \item Personal Rights Protection
    \end{itemize}
\end{itemize}

\subsubsection{Rationale for Grouping}
\begin{itemize}
    \item Facilitates identification of thematic strengths and weaknesses;
    \item Enables multi-level readiness assessment -- at the criterion, mid-group, and main group levels.
\end{itemize}

\subsubsection{Scoring System}
\begin{itemize}
    \item Sub-criteria: scored 0 or 1 (No / Yes -- information present);
    \item Mid-groups: scored from 0 to 3 (ranging from no disclosure to comprehensive, systematic, and quantitative reporting).
    \item Each company’s total score reflects a relative level of SSR readiness.
\end{itemize}

\subsection{Data Analysis}
\subsubsection{Analytical Steps}
\begin{itemize}
    \item Score each company based on the framework;
    \item Aggregate scores by thematic group;
    \item Compare across companies to identify common patterns and notable differences.
\end{itemize}

\subsubsection{Descriptive Statistical Techniques}
\begin{itemize}
    \item Means, frequencies, and standard deviations for total and group-level scores;
    \item Visualization tools (bar charts, radar charts, etc.);
    \item Identify salient factors and trends in SSR readiness.
\end{itemize}

\section{Research Ethics and Methodological Limitations}
\subsection{Research Ethics}
\begin{itemize}
    \item Only publicly available and legally accessible data from official sources is used;
    \item No individual company is evaluated or criticized---the goal is to provide an industry-level analysis;
    \item Adheres to academic integrity and protection of potentially sensitive data (if applicable).
\end{itemize}

\subsection{Limitations}
\begin{itemize}
    \item The sample is limited to 30 companies and may not represent the entire software industry;
    \item Data quality is dependent on the extent and clarity of each company’s public disclosures;
    \item The framework remains in a preliminary stage and has not been internally or externally validated.
\end{itemize}