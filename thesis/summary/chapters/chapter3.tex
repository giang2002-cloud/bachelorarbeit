\chapter{RESEARCH METHODOLOGY}

\section{Research Design and Methodological Approach to Readiness Assessment}

\subsubsection{Research Design}
The study adopts an exploratory mixed-methods approach, combining qualitative content analysis 
and descriptive statistics. 
A scoring-based framework for assessing SSR readiness was developed from ESRS and GRI.

\subsubsection{Methodological Approach}
The research proceeds in three phases:
\begin{enumerate}
    \item Developing the framework, consisting of 7 main groups and 246 sub-criteria;
    \item Collecting data from 30 European software companies in 2023;
    \item Descriptive analysis to assess levels of readiness and identify key patterns.
\end{enumerate}

\section{SSR Readiness Framework: Development and Evaluation Strategy}
\subsubsection{Reference Standards}
\begin{itemize}
    \item ESRS S1 -- Own Workforce;
    \item GRI 2 -- General Disclosures, GRI 401--406;
\end{itemize}

\subsubsection{Framework Structure}
\begin{itemize}
    \item A total of 246 sub-criteria (reporting items), partially derived from ESRS/GRI standards, and partially adapted to the 
    software sector context.
    \item 7 main indicator groups based on SRN Framework, each containing 2--3 mid-groups, reflecting specific social dimensions including:
    \begin{enumerate}
        \item Workforce Characteristics
        \item Collective bargaining and social dialogue
        \item Compensation
        \item Training
        \item Health and Safety
        \item Work Life Balance
        \item Human Rights
    \end{enumerate}
    \item A brief description is given on 
    how the full list of 246 sub-criteria was organized into mid-level thematic groups (mid-groups) for analysis. 
\end{itemize}

\subsubsection{Rationale for Grouping}
The grouping rationale is to identify thematic strengths and weaknesses and to facilitate 
multi-level readiness assessment.

\subsubsection{Scoring System}
    \begin{itemize}
        \item Sub-criteria: scored 0 or 1 (No / Yes -- information present);
        \item Mid-groups: scored from 0 to 3:
        
        \hspace{1cm} 0 = Not ready - No information reported;
        
        \hspace{1cm} 1 = Initial stage - Criteria mentioned but not quantified Criteria mentioned but not quantified with description 
        or assumption or qualitative disclosure;
        
        \hspace{1cm} 2 = Partially ready - Specific data reported for one dimension (e.g., gender);
        
        \hspace{1cm} 3 = Fully ready - Comprehensive reporting, including bargaining across two or 

        \hspace{1cm} more dimensions

        \item Main groups: average of mid-group scores.
        \item Total score: average of all 7 main group scores.
        \item Each company's total score reflects a relative level of SSR readiness.
    \end{itemize}

\subsubsection{Data Analysis}
The research includes descriptive statistics (mean, SD, frequency), thematic comparison, and radar/bar chart visualizations.


