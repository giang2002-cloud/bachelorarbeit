\chapter{PHƯƠNG PHÁP NGHIÊN CỨU}

\section{Thiết kế nghiên cứu và cách tiếp cận phương pháp luận đối với đánh giá mức độ sẵn sàng}

\subsubsection{Thiết kế nghiên cứu}
Nghiên cứu này áp dụng phương pháp hỗn hợp khám phá, kết hợp giữa phân tích nội dung định tính và thống kê mô tả.
Một khung đánh giá mức độ sẵn sàng báo cáo bền vững xã hội (SSR) dựa trên hệ thống chấm điểm đã được xây dựng từ ESRS và GRI.

\subsubsection{Cách tiếp cận phương pháp luận}
Nghiên cứu tiến hành qua ba giai đoạn:
\begin{enumerate}
\item Xây dựng khung đánh giá, bao gồm 7 nhóm chính và 246 tiêu chí phụ;
\item Thu thập dữ liệu từ 30 công ty phần mềm tại Châu Âu và Hoa Kỳ trong năm 2023;
\item Phân tích mô tả nhằm đánh giá mức độ sẵn sàng và xác định các mẫu hình chính.
\end{enumerate}

\section{Khung đánh giá mức độ sẵn sàng SSR: Chiến lược phát triển và đánh giá}

\subsubsection{Chuẩn tham chiếu}
\begin{itemize}
\item ESRS S1 – Own Workforce;
\item GRI 2 – General Disclosures, GRI 401–406;
\end{itemize}

\subsubsection{Cấu trúc khung}
\begin{itemize}
\item Tổng cộng 246 tiêu chí phụ (các mục báo cáo), được xây dựng một phần từ ESRS/GRI và một phần điều chỉnh theo bối cảnh ngành dịch vụ phần mềm.
\item 7 nhóm chỉ số chính dựa trên Khung SRN, mỗi nhóm bao gồm 2–3 nhóm trung gian, phản ánh các khía cạnh xã hội cụ thể, bao gồm:
\begin{enumerate}
\item Đặc điểm lực lượng lao động
\item Thương lượng tập thể và đối thoại xã hội
\item Đãi ngộ
\item Đào tạo
\item Sức khỏe và an toàn
\item Cân bằng công việc – cuộc sống
\item Nhân quyền
\end{enumerate}
\item Một mô tả ngắn gọn được trình bày về cách toàn bộ 246 tiêu chí phụ được tổ chức thành các nhóm chủ đề trung gian để phục vụ phân tích.
\end{itemize}

\subsubsection{Cơ sở lý luận cho việc phân nhóm}
Mục đích phân nhóm là nhằm xác định điểm mạnh và điểm yếu theo từng chủ đề, đồng thời hỗ trợ đánh giá mức độ sẵn sàng đa cấp độ.

\subsubsection{Hệ thống chấm điểm}
\begin{itemize}
\item Tiêu chí phụ: chấm 0 hoặc 1 (Không / Có – có thông tin báo cáo);
\item Nhóm trung gian: chấm từ 0 đến 3:
    \hspace{1cm} 0 = Chưa sẵn sàng – Không có thông tin báo cáo;  
    
    \hspace{1cm} 1 = Giai đoạn khởi đầu – Tiêu chí được đề cập nhưng không định lượng, chỉ mô tả hoặc giả định mang tính định tính;  
    
    \hspace{1cm} 2 = Sẵn sàng một phần – Có dữ liệu cụ thể cho một khía cạnh (ví dụ: giới tính);  
    
    \hspace{1cm} 3 = Hoàn toàn sẵn sàng – Báo cáo đầy đủ, bao gồm dữ liệu cho từ hai khía cạnh trở lên.  

\item Nhóm chính: giá trị trung bình của các nhóm trung gian.  
\item Tổng điểm: trung bình của cả 7 nhóm chính.  
\item Tổng điểm của mỗi công ty phản ánh mức độ sẵn sàng SSR tương đối.  

\end{itemize}

\subsubsection{Phân tích dữ liệu}
Nghiên cứu bao gồm phân tích thống kê mô tả (giá trị trung bình, độ lệch chuẩn, tần suất), so sánh theo chủ đề, và trực quan hóa bằng biểu đồ radar và biểu đồ cột.


