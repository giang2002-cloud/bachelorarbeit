\chapter{RESEARCH METHODOLOGY}

\section{Research Design and Methodological Approach for Social Sustainability Readiness}

\begin{itemize}
    \item The study follows an exploratory, mixed-methods approach:
    \begin{itemize}
        \item Qualitative content analysis of company reports;
        \item Quantitative descriptive statistics;
        \item Framework design aligned with Design Science Research (DSR) principles.
    \end{itemize}
\end{itemize}

\begin{itemize}
    \item Three main phases of the research:
    \begin{enumerate}
        \item Developing the framework, consisting of 7 main groups and 246 sub-criteria;
        \item Collecting data from 30 European software companies (2022-2023);
        \item Descriptive analysis to assess levels of readiness and identify key patterns.
    \end{enumerate}
\end{itemize}



\section{Assessment Framework and Data Analysis}
\subsection{Developing the SSR Readiness Framework}
\subsubsection{Reference Standards}
\begin{itemize}
    \item ESRS S1 -- Own Workforce;
    \item GRI 2 -- General Disclosures, GRI 401--406;
\end{itemize}

\subsubsection{Framework Structure}
\begin{itemize}
    \item A total of 246 sub-criteria (reporting items), partially derived from ESRS/GRI standards, and partially adapted to the 
    software sector context.
    \item 7 main indicator groups based on SRN Framework, each containing 2--3 mid-groups, reflecting specific social dimensions including:
    \begin{enumerate}
        \item Workforce Characteristics
        \item Collective bargaining and social dialogue
        \item Compensation
        \item Training
        \item Health and Safety
        \item Work Life Balance
        \item Human Rights
    \end{enumerate}
    \item A brief description is given on 
    how the full list of 246 sub-criteria was organized into mid-level thematic groups (mid-groups) for analysis. 
\end{itemize}

\subsubsection{Rationale for Grouping}
\begin{itemize}
    \item Facilitates identification of thematic strengths and weaknesses;
    \item Enables multi-level readiness assessment -- at the criterion, mid-group, and main group levels.
\end{itemize}

\subsubsection{Scoring System}
\begin{itemize}
    \item Sub-criteria: scored 0 or 1 (No / Yes -- information present);
    \item Mid-groups: scored from 0 to 3:
    \item \begin{itemize}
        \item 0 = No information;
        \item 1 = Criteria are mentioned but not quantified;
        \item 2 = Specific data or bargaining is reported for one dimension (e.g., gender);
        \item 3 = Comprehensive reporting, including bargaining across two or more dimensions
    \end{itemize}
    \item Main groups: average of mid-group scores.
    \item Total score: average of all 7 main group scores.
    \item Each company's total score reflects a relative level of SSR readiness.
\end{itemize}

\subsection{Data Analysis}
Includes descriptive statistics (mean, SD, frequency), thematic comparison, and radar/bar chart visualizations.

\section{Research Ethics and Methodological Limitations}
\begin{itemize}
    \item Uses only public data, no company is individually criticized.
    \item Sample may not fully represent the software sector.
    \item Disclosure depth varies across firms; framework is still exploratory and unvalidated.
\end{itemize}
