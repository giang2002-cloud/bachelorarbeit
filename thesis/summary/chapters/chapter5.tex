\chapter{THẢO LUẬN VÀ HÀM Ý}

\section{Hàm ý đối với thực tiễn}

\subsubsection{Đối với doanh nghiệp}
\begin{itemize}
\item SSR không nên chỉ được coi là tuân thủ quy định, mà còn là một khoản đầu tư chiến lược cho sự phát triển bền vững dài hạn.
\end{itemize}

\subsubsection{Đối với nhà hoạch định chính sách}
\begin{itemize}
\item Các nhà hoạch định chính sách cần khuyến khích SSR thông qua hướng dẫn đặc thù ngành, đào tạo, ưu đãi tài chính, và các chuẩn dữ liệu mở (ví dụ: SRN) nhằm thúc đẩy tính minh bạch.
\end{itemize}

\subsubsection{Đối với tổ chức ban hành chuẩn mực và các cơ quan xếp hạng}
\begin{itemize}
\item Các chỉ số xã hội cần được điều chỉnh phù hợp với ngành công nghiệp số, với ngưỡng trọng yếu rõ ràng hơn để giảm bớt sự mơ hồ.
\end{itemize}

\section{Đóng góp học thuật và phản tư lý thuyết}

\subsubsection{Đóng góp}
Nghiên cứu này mở rộng văn liệu ESG sang lĩnh vực dịch vụ phần mềm – một ngành còn ít được nghiên cứu, đề xuất một khung đánh giá mức độ sẵn sàng SSR có thể chuyển giao, đồng thời chứng minh cách Lý thuyết thể chế, Lý thuyết các bên liên quan và Quan điểm dựa trên nguồn lực (RBV) có thể kết hợp để giải thích sự khác biệt trong hành vi báo cáo.

\subsubsection{Phản tư phê phán}
Mặc dù nghiên cứu này tích hợp nhiều lý thuyết tổ chức để giải thích mức độ sẵn sàng SSR, song việc áp dụng các lý thuyết đó không phải lúc nào cũng phù hợp trong mọi trường hợp.