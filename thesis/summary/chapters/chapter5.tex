\chapter{DISCUSSION ON EMPIRICAL RESULTS}

\section{Summary of Key Findings}
Brief recap of key results from Chapter 4:
\begin{itemize}
    \item Average SSR readiness among the 30 companies
    \item Strongest and weakest thematic categories
    \item Differences across company groups (by size, geography, etc.)
    \item Unexpected or counterintuitive observations
\end{itemize}

\section{Theoretical Interpretation of Drivers and Barriers}
\subsection{Drivers of Social Sustainability Reporting Readiness}

\begin{table}[H]
    \centering
    \caption{Drivers of SSR Reporting and Theoretical Interpretation}
    \begin{tabular}{p{6cm}|p{8cm}}
        \subsubsection{Driver} & \subsubsection{Theoretical Interpretation} \\
        \hline
        Regulatory pressure from CSRD, EFRAG, SEC & Institutional Theory: Coercive pressure driving behavioral change \\
        \hline
        Demands from customers and large investors & Stakeholder Theory: Stakeholder expectations incentivize transparency and reporting quality \\
        \hline
        Talent shortage $\rightarrow$ SSR used for employer branding & Stakeholder \& RBV: SSR becomes a competitive advantage in recruitment and reputation \\
        \hline
        Large firms with clear ESG leadership & RBV: Strong organizational capabilities, leadership, and ESG integration into strategy    
    \end{tabular}
\end{table}

\subsection{Barriers of Social Sustainability Reporting Readiness}

\begin{table}[H]
    \centering
    \caption{Industry-Specific Challenges and Theoretical Interpretation}
    \begin{tabular}{p{6cm}|p{8cm}}
        \subsubsection{Challenge} & \subsubsection{Theoretical Interpretation} \\
        \hline
        Difficulty in collecting and standardizing non-financial data & RBV: Reflects lack of systems, tools, and personnel—organizational capabilities not yet developed \\
        \hline
        Informal labor, globalization, remote work & Institutional Theory: Fragmented settings weaken legal coherence and coercive pressure \\
        \hline
        Lack of sector-specific social standards & Institutional Theory: Normative pressure is underdeveloped; no established "social norms" for the sector \\
        \hline
        SMEs lack ESG budget/personnel & RBV: SMEs often lack the strategic resources to build internal reporting capabilities
    \end{tabular}
\end{table}

\section{Implications for Practices}
\subsubsection{For Businesses}
\begin{itemize}
    \item SSR should be viewed not merely as compliance but as a strategic investment in long-term
\end{itemize}

\subsubsection{For Policymakers}
\begin{itemize}
    \item Encourage SSR through sector-specific guidance, training, financial incentives, and open data standards (e.g., SRN) to support transparency.
\end{itemize}

\subsubsection{For Standard-Setters and Rating Agencies}
\begin{itemize}
    \item Tailor social indicators to digital industries and clarify materiality thresholds to reduce reporting ambiguity.
\end{itemize}

\section{Academic Contributions}
\begin{itemize}
    \item Extends ESG research into the software industry, a largely underexplored area
    \item Develops a transferable SSR readiness framework applicable to other service sectors
    \item Integrates three organizational theories in a complementary way to explain ESG reporting behavior
\end{itemize}