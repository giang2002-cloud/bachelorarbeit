\chapter{DISCUSSION AND IMPLICATIONS}


\section{Implications for Practice}
\subsubsection{For Businesses}
\begin{itemize}
    \item SSR should be viewed not merely as compliance but as a strategic investment in long-term
\end{itemize}

\subsubsection{For Policymakers}
\begin{itemize}
    \item Policymakers should encourage SSR through sector-specific guidance, training, financial incentives, 
    and open data standards (e.g., SRN) to support transparency.
\end{itemize}

\subsubsection{For Standard-Setters and Rating Agencies}
\begin{itemize}
    \item Social indicators should be tailored to digital industries,
    with clearer materiality thresholds to reduce ambiguity.
\end{itemize}

\section{Academic Contributions and Theoretical Reflections}
\subsubsection{Contributions}
This study expands ESG literature to the under-researched software services sector, 
proposes a transferable SSR readiness framework, and demonstrates how 
Institutional Theory, Stakeholder Theory, and RBV can jointly explain differences in reporting behavior

\subsubsection{Critical Reflection}
While this study integrates multiple organizational theories to explain SSR readiness, 
their application is not universally appropriate in all cases.
