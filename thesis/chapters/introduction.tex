% Chapter 1
\chapter{INTRODUCTION}

You have to \textbf{submit the Bachelor's thesis} on the submisson date to the department. This is the introduction chapter. Write your content here.

It should provide an overview of your research topic, the problem statement, and the objectives of your study. Discuss the significance of your research and how it contributes to the field.
\section{Background}
Provide background information on your research topic. Explain the context and relevance of your study. This section
asd should help the reader understand why your research is important and what led you to choose this topic.
\section{Problem Statement}
Clearly articulate the problem that your research aims to address. Discuss the specific issues or challenges that exist in the current literature or practice. Explain why these problems are significant and worth investigating.
\section{Objectives}
The choice of 2023 as the temporal focus of this study is closely linked to the institutional context. 
While the Corporate Sustainability Reporting Directive (CSRD) formally entered into force in January 2023, 
companies are required to disclose in line with the European Sustainability Reporting Standards (ESRS) for 
fiscal years starting on or after January 2024. Thus, 2023 represented a preparatory phase in which firms 
faced rising institutional pressure, whereas 2024 marks the first mandatory reporting cycle. 
This transition reflects the increasing coercive institutional forces that shape companies’ readiness 
for social sustainability reporting.

Social pillars include employment, labor relations, diversity,
occupational health and safety, and human rights (Fiechter 2022; Edge 2022).