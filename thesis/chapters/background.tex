\chapter{Literature Review and Theoretical Framework}
\label{chap:background}
\section{Prior Literature on Social Sustainability Reporting}

Sustainability reporting can be understood under various definitions, 
including corporate social responsibility (CSR), extended external reporting, 
and ESG reporting \parencite{Edge2022}. As defined by \parencite{Rasche2017}, CSR is the integration of social, 
environmental, ethical, and philanthropic responsibilities into business practice. 
Meanwhile, the concept of ESG stands for Environmental, Social, and Governance \parencite{UNGlobal2024}.
According to \parencite{Krivogorsky2024}, three components of sustainability reporting 
can be associated with three main drivers of sustainability, namely the reporting entity, 
society and ecology. Across these approaches, the social dimension consistently emerges 
as a core aspect, covering issues such as employment, working conditions 
and human rights \parencite{Fiechter2022, Morais2018}.

Several regulations and standards have been developed to provide a basis for social sustainability reporting.
Among these, the Global Reporting Initiative (GRI) is regarded as the most widely used framework 
by companies worldwide \parencite{Bais2024, vanOorschot2024}. 
According to GRI 401-406, companies disclose a wide range of social aspects, such as employment,
workplace conditions, equality issues and non-discrimination.
Marking a significant step forward, the European Sustainability Reporting Standards (ESRS) 
broaden this scope and make disclosure mandatory. The European Commission adopted the ESRS 
in July 2023 under the Corporate Sustainability Reporting Directive (CSRD). These standards 
oblige companies to report on impacts concerning the workforce (S1), value chain workers (S2), 
affected communities (S3), and consumers and end-users (S4).

Despite these regulatory developments, prior studies show that the social pillar remains underdeveloped 
within sustainability reporting practices. According to \textcite{Heldal2024}, the social dimension 
has not received adequate attention. This can be explained by \textcite{Sharma2024}, who argue that
social data is a critical but often neglected component of sustainability reporting. Similarly, 
\textcite{Morais2018} point out that the social pillar remains vague, difficult to quantify, 
and less prioritized than environmental issues. This view is reinforced by \textcite{Berg2022}, 
who demonstrate lower inter-rater consensus on social indicators compared to environmental ones.

\section{Prior Literature on SSR Readiness}

Most prior research examines the extent to which firms comply with the GRI Standards 
when issuing sustainability reports \parencite{Fonseca2014, Vigneau2015}. 
This emphasis is to be expected, as the GRI has represented a pioneering initiative 
in defining a common foundation that enables companies to develop stakeholder-oriented 
sustainability reports \parencite{Carungu2025, Dumay2010}. In 2023, GRI and EFRAG released a draft 
Interoperability Index to help preparers align GRI Standards with ESRS disclosure 
requirements in detail \parencite{GRIEFRAG2024}. Given this close interrelation, the present study begins 
by reviewing prior literature on GRI-based compliance in sustainability (CSR) reporting, 
which provide a basis for the analysis of SSR readiness.

For example, \textcite{Ehnert2016} employed the GRI guidelines as an analytical framework 
to assess aspects of Sustainable Human Resource Management (HRM). Their study shows that 
companies tend to report more extensively on their internal workforce compared to their 
external workforce \parencite{Ehnert2016}. Meanwhile, issues related to collective labor 
rights are often underreported, as firms exploit gaps in the GRI standards—particularly 
in disclosures 402 and 407 \parencite{Waas2021}. Consistently, \textcite{Parsa2018} found 
that transnational corporations failed to comply with GRI's "labour" and "human rights"
reporting guidelines. Similarly, the quality of information disclosed on occupational 
health, safety (OHS), and employee well-being remains generally low \parencite{Mariappanadar2022}. 
Specifically, \textcite{Mariappanadar2022} highlight that firms often provide only generic and 
anecdotal disclosures to meet GRI requirements, rarely translating their claims into measurable
outcomes. In line with these findings, \textcite{Chauvey2015} also demonstrate that the quality
of human resource-related reporting remains relatively poor.

At present, the introduction of the ESRS shifts attention toward firms' readiness to 
meet more comprehensive and mandatory reporting requirements. However, most prior studies 
have addressed overall sustainability or CSR reporting readiness,
but have not examined social pillar readiness in isolation. Current research provides only 
limited empirical insights into the extent to which companies are adjusting to the new 
sustainability reporting requirements. Notable exceptions are \textcite{Nicolo2025}, who examined early compliance with 
ESRS disclosure requirements among Italian listed firms and identified key firm-level drivers of 
adaptation, and \textcite{Filho2025}, who assessed the readiness of 20 firms drawn from a cross-sectoral 
sample of EU companies. \textcite{Nicolo2025} reports relatively high levels of early compliance 
with ESRS requirements among firms falling within the scope of the Omnibus Package. Whereas, 
\textcite{Raimo2025} documents relatively low overall levels of pre-compliance, 
particularly in integrated reports. \textcite{Filho2025} and \textcite{Montero2025} likewise reveal 
that disclosure, especially among SMEs, remains limited, due to the absence of concrete metrics, 
verification mechanisms, and external assurance. Overall, prior studies have focused on 
sustainability reporting in general and highlighted industry environmental sensitivity 
as a key determinant of readiness \parencite{Filho2025, Raimo2025, Nicolo2025, Montero2025}.
Specifically, these studies have focused on environmentally intensive sectors such as oil and 
energy \parencite{Filho2025, Raimo2025, Nicolo2025}, while the software industry has received 
only limited attention, typically being considered within broader technology-oriented 
classifications \parencite{Montero2025}.

\section{Sectoral Characteristics and Reporting Practices in Software Services}

The software services sector is a service-oriented industry that primarily provides intangible assets 
in the form of digital goods \parencite{Buxmann2015}. Within this sector, the workforce plays a particularly 
crucial role \parencite{Buxmann2015}. Investment in employee capacity development through training programs 
and certification initiatives has been shown to positively impact company performance in the global IT 
services industry \parencite{Chatterjee2017}. Furthermore, \textcite{Nowak2000} emphasize that intellectual 
capital constitutes a core value of the software industry. This highlights why the software services sector 
has drawn significant attention in research on CSR reporting in general \parencite{Holder-Webb2009, Kim2018}.

To date, there remains a lack of empirical research specifically addressing the social pillar of CSR reporting 
within the software industry. For example, \textcite{Holder-Webb2009} examined software companies within a sample 
of 50 firms across multiple industries in the United States. The results indicate that employment-related issues 
as well as health and safety are relatively well addressed in the software sector \parencite{Holder-Webb2009}. 
In contrast, other social aspects, such as human rights and community engagement, are not reported in a 
comprehensive manner \parencite{Holder-Webb2009}.

Moreover, the software industry experiences rapid changes in technology and business models \parencite{Buxmann2015}.
Such continuous changes require companies to rapidly adapt to new regulations, such as the ESRS. 




Reporting on key aspects such as work environment, work-life balance, 
and training is often inconsistent and rarely comprehensive \parencite{Greig2021}. Similarly,
disclosures on gender diversity were limited and linked to worse gender pay gaps,
suggesting a reputational motive \parencite{Huang2022}.


Prior studies have highlighted 
persistent weaknesses in CSR report quality \parencite{DiChiacchio2024}, largely due to the scarcity of 
quantitative and monetary data \parencite{Michelon2015}.










\section{Theoretical frameworks}

This study employs theoretical frameworks to explain the current level 
of social sustainability reporting readiness in the software services sector 
and to interpret the drivers and barriers shaping firms' preparedness.
Prior studies have highlighted the importance of theoretical perspectives 
in the study of sustainability reporting \parencite{Gesso2023, Rezaee2016, Lozano2015}.
Within the wide range of theories applied in this field, Institutional Theory
and Legitimacy Theory are among those frequently employed.

\subsection{Institutional Theory}
Institutional Theory is widely applied in sustainability 
reporting research \parencite{Campbell2007}. This theory explains 
how organizations adapt to social, political, and economic environments 
through institutionalized practices 
\parencite{Meyer1977, DiMaggio1983}.
It emphasizes how coercive, normative, and mimetic pressures shape organizational behavior.


