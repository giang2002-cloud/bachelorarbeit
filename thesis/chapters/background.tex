\chapter{Literature Review and Theoretical Framework}
\label{chap:background}
\section{Literature Review}
\subsection{Prior Literature on Social Sustainability Reporting}

Sustainability reporting can be understood under various definitions, 
including corporate social responsibility (CSR), extended external reporting, 
and ESG reporting (Edge 2022; Fiechter 2022). As defined by Rasche (2017), CSR is the integration of social, 
environmental, ethical, and philanthropic responsibilities into business practice. 
Similarly, the ESG framework focuses on environmental, social, and governance. 
Across these approaches, the social dimension consistently emerges as a core aspect,
covering issues such as labor relations, diversity and human rights (Fiechter 2022; Edge 2022).

Several regulations and standards have been developed to provide a basis for social sustainability reporting.
Among these, the Global Reporting Initiative (GRI) is regarded as the most widely used framework 
by companies worldwide (Bais, 2024; van Oorschot, 2024). 
According to GRI 401-406, companies disclose social aspects covering employment, 
labor management relations, occupational health and safety, training and education, 
diversity and equal opportunity, and non-discrimination. 
Marking a significant step forward, the European Sustainability Reporting Standards (ESRS) 
broaden this scope and make disclosure mandatory. The European Commission adopted the ESRS 
in July 2023 under the Corporate Sustainability Reporting Directive (CSRD). These standards 
oblige companies to report on impacts concerning the workforce (S1), value chain workers (S2), 
affected communities (S3), and consumers and end-users (S4). Morais and Silvestre (2018) 
emphasize the need for a comprehensive approach to social sustainability reporting, highlighting the interconnectedness of these stakeholder groups.

Despite these regulatory developments, the literature indicates difficulties in the implementation 
of social pillar within sustainability reporting practices. According to Heldal et al. (2024), the social dimension 
has not received adequate attention. This can be explained by the findings of Morais and Silvestre (2018), 
who argue that the social pillar remains vague, difficult to quantify, and less prioritized than 
environmental issues. This view is reinforced by Berg et al. (2022), who demonstrate lower inter-rater 
consensus on social indicators compared to environmental ones. Key own-workforce aspects such as training and development, 
worklife balance, respect and inclusion, remuneration and benefits, and worker participation receive
limited coverage (Greig et al. 2021). While many large firms disclosed gender diversity data,
such disclosures were often limited and linked to worse gender pay gaps, suggesting a reputational motive 
(Huang and Lu 2022). 

\subsection{Prior Literature on Readiness}

In general, sustainability reports tend to convey 
predominantly qualitative information, with only limited quantitative or monetary data (Michelon 2014, ). 


\section{Theoretical frameworks}

This study employs theoretical frameworks to explain the current level 
of social sustainability reporting readiness in the software services sector 
and to interpret the drivers and barriers shaping firms' preparedness.
Prior studies have highlighted the importance of theoretical perspectives 
in the study of sustainability reporting (Gesso 2023; Rezaee 2016; Lozano 2015).
Within the wide range of theories applied in this field, Institutional Theory
and Legitimacy Theory are among those frequently employed.

\subsection{Institutional Theory}
Institutional Theory is widely applied in sustainability 
reporting research (Campbell, 2007). This theory explains 
how organizations adapt to social, political, and economic environments 
through institutionalized practices 
(Meyer and Rowan, 1977; DiMaggio and Powell, 1983).
It emphasizes how coercive, normative, and mimetic pressures shape organizational behavior.


