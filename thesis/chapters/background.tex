\chapter{Literature Review and Theoretical Framework}
\label{chap:background}
\section{Prior Literature on Social Sustainability Reporting}

Sustainability reporting can be understood under various definitions, 
including corporate social responsibility (CSR), extended external reporting, 
and ESG reporting \parencite{Edge2022}. As defined by \parencite{Rasche2017}, CSR is the integration of social, 
environmental, ethical, and philanthropic responsibilities into business practice. 
Meanwhile, the concept of ESG stands for Environmental, Social, and Governance \parencite{UNGlobal2004}.
According to \parencite{Krivogorsky2024}, three components of sustainability reporting 
can be associated with three main drivers of sustainability, namely the reporting entity, 
society and ecology. Across these approaches, the social dimension consistently emerges 
as a core aspect \parencite{Rasche2017, UNGlobal2024, Krivogorsky2024}, covering issues 
such as employment, working conditions and human rights \parencite{Fiechter2022, Morais2018}.

Prior studies relates to the social pillar within sustainability reporting practices. Social issues 
appear to attract strong interest from organizations and their customers, however, this interest does 
not necessarily translate into the inclusion of social objectives in sustainability goals \parencite{Heldal2024}. 
This can be explained by \textcite{Sharma2024}, who argue that
social data is a critical but often neglected component of sustainability reporting. Similarly, 
\textcite{Morais2018} point out that the social pillar remains vague, difficult to quantify, 
and less prioritized than environmental issues. This view is reinforced by \textcite{Berg2022}, 
who demonstrate lower inter-rater consensus on social indicators compared to environmental ones.

Several regulations and standards have been developed to provide a basis for social sustainability reporting.
Among these, the Global Reporting Initiative (GRI) is regarded as the most widely used framework 
by companies worldwide \parencite{Bais2024, vanOorschot2024}. 
According to GRI 401-406, companies disclose a wide range of social aspects, such as employment,
workplace conditions, equality issues and non-discrimination.
Marking a significant step forward, the European Sustainability Reporting Standards (ESRS) 
broaden this scope and make disclosure mandatory. The European Commission adopted the ESRS 
in July 2023 under the Corporate Sustainability Reporting Directive (CSRD). These standards 
oblige companies to report on impacts concerning the workforce (S1), value chain workers (S2), 
affected communities (S3), and consumers and end-users (S4).

\section{Prior Literature on SSR Readiness}

Most prior research examines the extent to which firms comply with the GRI Standards 
when issuing sustainability reports \parencite{Fonseca2014, Vigneau2015}. 
This emphasis is to be expected, as the GRI has represented a pioneering initiative 
in defining a common foundation that enables companies to develop stakeholder-oriented 
sustainability reports \parencite{Carungu2025, Dumay2010}. In 2023, GRI and EFRAG released a draft 
Interoperability Index to help preparers align GRI Standards with ESRS disclosure 
requirements in detail \parencite{GRIEFRAG2024}. Given this close interrelation, this study begins 
by reviewing prior literature on GRI-based compliance in sustainability (CSR) reporting, 
which provide a basis for the analysis of SSR readiness.

For example, \textcite{Ehnert2016} employed the GRI guidelines as an analytical framework 
to assess aspects of Sustainable Human Resource Management (HRM). The study shows that 
companies tend to report more extensively on their internal workforce compared to their 
external workforce \parencite{Ehnert2016}. Meanwhile, issues related to collective labor 
rights are often underreported, as firms exploit gaps in the GRI standards—particularly 
in disclosures 402 and 407 \parencite{Waas2021}. Consistently, \textcite{Parsa2018} found 
that transnational corporations failed to comply with GRI's "labour" and "human rights"
reporting guidelines. Similarly, the quality of information disclosed on occupational 
health, safety (OHS), and employee well-being remains generally low \parencite{Mariappanadar2022}. 
Specifically, \textcite{Mariappanadar2022} highlight that firms often provide only generic and 
anecdotal disclosures to meet GRI requirements, rarely translating their claims into measurable
outcomes. In line with these findings, \textcite{Chauvey2015} also demonstrate that the quality
of human resource-related reporting remains relatively poor.

At present, the introduction of the ESRS shifts attention toward firms' readiness to 
meet more comprehensive and mandatory reporting requirements. However, most prior studies 
have addressed overall sustainability or CSR reporting readiness,
but have not examined social pillar readiness in isolation. Current research provides only 
limited empirical insights into the extent to which companies are adjusting to the new 
sustainability reporting requirements. Notable exceptions are \textcite{Nicolo2025}, who examined early compliance with 
ESRS disclosure requirements among Italian listed firms and identified key firm-level drivers of 
adaptation, and \textcite{Filho2025}, who assessed the readiness of 20 firms drawn from a cross-sectoral 
sample of EU companies. \textcite{Nicolo2025} reports relatively high levels of early compliance 
with ESRS requirements among firms falling within the scope of the Omnibus Package. Whereas, 
\textcite{Raimo2025} documents relatively low overall levels of pre-compliance, 
particularly in integrated reports. \textcite{Filho2025} and \textcite{Montero2025} likewise reveal 
that disclosure, especially among SMEs, remains limited, due to the absence of concrete metrics, 
verification mechanisms, and external assurance. Overall, prior studies have focused on 
sustainability reporting in general and highlighted industry environmental sensitivity 
as a key determinant of readiness \parencite{Filho2025, Raimo2025, Nicolo2025, Montero2025}.
Specifically, these studies have focused on environmentally intensive sectors such as oil and 
energy \parencite{Filho2025, Raimo2025, Nicolo2025}, while techonology industry has received 
only limited attention \parencite{Montero2025}.

\section{Theoretical frameworks}

This study employs theoretical frameworks to interpret the drivers and barriers shaping firms' preparedness.
Prior studies have highlighted the importance of theoretical perspectives 
in the study of sustainability reporting \parencite{Gesso2023, Rezaee2016, Lozano2015}.
Within the wide range of theories applied in this field, Institutional Theory
and Legitimacy Theory are among those frequently employed.

\subsection{Institutional Theory}
Institutional Theory is widely applied in sustainability 
reporting research (e.g., \citeauthor{Campbell2007} \citeyear{Campbell2007}; \citeauthor{Nikolaeva2010} \citeyear{Nikolaeva2010};
\citeauthor{Shabana2017} \citeyear{Shabana2017}). This theory explains 
how organizations adapt to social, political, and economic environments 
through institutionalized practices 
\parencite{Meyer1977, DiMaggio1983}.
It emphasizes how coercive, normative, and mimetic pressures shape organizational behavior.

Coercive pressures stem from formal laws, regulations, and mandates, as well as from powerful stakeholders, 
such as governments, regulatory agencies, or large customers. For instance, the European Union's Corporate 
Sustainability Reporting Directive (CSRD) and the accompanying European Sustainability Reporting Standards 
(ESRS) impose mandatory disclosure requirements on social, environmental, and governance dimensions. 
Firms, including those in the software services sector, face coercive pressure to comply with these standards, 
as non-compliance could result in legal penalties, reputational damage, or restricted access to markets.

Normative pressures arise from professional standards, ethical expectations, and industry-specific norms. 
Firms experience these pressures when stakeholders, including investors, clients, employees, and industry 
associations expect organizations to follow best practices in social responsibility and sustainability 
reporting. In the context of software services, normative pressures manifest through expectations for 
transparent disclosure of workforce-related metrics such as employee training, diversity and inclusion, 
labor conditions, and occupational health and safety. Professional networks and industry consortia often 
establish benchmarks or guidelines, which further reinforce these normative pressures.

Mimetic pressures occur when organizations emulate peers, competitors, or leading firms in uncertain or 
ambiguous environments. In sectors characterized by intangible assets and rapid technological change, 
such as software services, firms may lack clear guidelines on social reporting practices. As a result, 
they often look to industry leaders or larger firms with established CSR or ESG reporting mechanisms as models. 
Mimetic behavior helps firms reduce uncertainty and demonstrate alignment with perceived best practices, 
thereby reinforcing legitimacy among stakeholders.

Institutional theory has been widely applied to explain the adoption of corporate social responsibility 
(CSR) and sustainability reporting across industries. Previous studies suggest that organizations are 
motivated to adopt reporting practices not solely for intrinsic ethical reasons but to conform to institutional 
expectations and to signal legitimacy to stakeholders (Marquis et al., 2007; Campbell, 2007). Firms'readiness 
to implement social sustainability reporting frameworks, such as ESRS, can therefore be interpreted as a response 
to these institutional pressures. Adoption patterns may differ depending on firm size, ownership structure, 
international exposure, and prior experience with sustainability initiatives.

\subsection{Legitimacy Theory}

Legitimacy theory is considered one of the most successful approaches for explaining the content 
and extent of disclosed social information \parencite{Grey1995}. This theory originates from the concept of the social 
contract \parencite{Patten1991}. \textcite{Schoker1974} argued that “any social institution [including business] 
operates in society via a social contract, expressed or implied”. Meeting social objectives directly influences 
a business's survival and development \parencite{Schoker1974}. Therefore, organizations continuously strive to 
maintain congruence between their actions and the norms, values, and expectations of society \parencite{Suchman1995}. 
This alignment is considered by businesses as a "license to operate" \parencite{Demuijnck2016}. When an organization 
is perceived as legitimate, audiences tend to view it as not only more worthy, but also more meaningful, predictable, 
and trustworthy \parencite{Suchman1995}. Legitimacy is categorized into three principal forms: pragmatic, moral, and cognitive
\parencite{Suchman1995}. 

Pragmatic legitimacy depends on stakeholders' self-interest. When stakeholders perceive tangible 
and direct benefits, legitimacy is granted. Similar to pragmatic legitimacy, moral legitimacy relies on stakeholders' 
active support. It is determined by 
whether the company's actions are perceived as morally appropriate. Meanwhile, cognitive legitimacy is difficult 
for companies to regulate because of its inherent implicitness. When organizational practices are understandable, 
widely accepted and aligned with prevailing societal expectations, cognitive legitimacy is recognized \parencite{Suchmann1995}

Corporate disclosure can be used as a tool to shape how external stakeholders perceive the organization \parencite[p.~11]{Deegan2002}.


In the context of corporate social responsibility (CSR) and sustainability reporting, legitimacy theory provides 
a robust explanatory framework. Organizations may adopt CSR initiatives and disclose information about social, 
environmental, or governance performance to signal alignment with societal expectations, maintain stakeholder 
confidence, and preempt criticism or regulatory scrutiny. This signaling function is particularly relevant 
when organizational activities are visible to external stakeholders or when past incidents have raised concerns 
regarding ethical conduct, social responsibility, or environmental stewardship. By strategically managing legitimacy 
through disclosure, organizations aim to reduce potential reputational risks, reinforce stakeholder relationships, 
and demonstrate accountability.

Legitimacy theory also elucidates why organizations may engage in selective or symbolic reporting practices. 
When the cost of substantive social or environmental action is high, or when stakeholders’ expectations are 
ambiguous or inconsistent, firms may prioritize the appearance of responsible behavior over the implementation 
of substantive measures. This strategic use of reporting allows organizations to maintain perceived legitimacy 
without incurring the full costs associated with actual compliance or performance improvement. Such practices 
underscore the complex interplay between organizational behavior, societal norms, and stakeholder perceptions.

Academic research has highlighted the pervasive influence of legitimacy considerations across diverse sectors 
and organizational contexts. For instance, firms in high-visibility industries—such as finance, energy, and 
consumer goods—often engage in structured disclosure practices to demonstrate compliance with societal norms 
and to enhance their public image. Legitimacy theory explains not only the adoption of reporting mechanisms 
but also variations in reporting quality, transparency, and emphasis across different organizations, as firms 
calibrate their practices to match the expectations of key stakeholder groups.

Beyond the adoption of reporting practices, legitimacy theory emphasizes the dynamic nature of organizational 
legitimacy. Stakeholder expectations evolve over time, influenced by cultural, social, and regulatory shifts.
Consequently, organizations must continuously monitor external expectations, assess the alignment of their actions 
with these norms, and adjust their strategies accordingly. Failure to do so can result in legitimacy deficits, 
reputational damage, or loss of market opportunities. Conversely, proactive management of legitimacy can confer 
competitive advantages, strengthen organizational resilience, and foster stakeholder loyalty.




Building on the concept that CSR reporting serves as a legitimacy-seeking behavior (Author A, Year; Author B, Year), 
social sustainability reporting (SSR) can be interpreted as a proactive organizational effort to maintain or restore legitimacy.
From a pragmatic perspective, social sustainability reporting (SSR) concerning the organization's own workforce enables firms 
to respond to stakeholders' immediate interests. Such interests encompass employees' well-being, labor conditions, and diversity
considerations. Addressing these concerns may also affect investors, clients, and partners who prioritize responsible management 
of human capital. From a moral perspective, reporting on workforce-related practices signals that the organization acts ethically. 
It demonstrates that the organization fulfills its social responsibilities toward its employees.
From a cognitive perspective, transparent and consistent disclosure of workforce-related data makes reporting practices 
understandable. This approach helps such practices become taken-for-granted, thereby enhancing the perceived appropriateness 
of the organization's actions within society.


The application of legitimacy theory can provide a foundation for later application in the analysis of sector-specific drivers 
and challenges for social sustainability reporting readiness.



\section{Sectoral Characteristics and Reporting Practices in Software Services}

The software services sector is a service-oriented industry that primarily provides intangible assets 
in the form of digital goods \parencite{Buxmann2015}. Within this sector, the workforce plays a particularly 
crucial role \parencite{Buxmann2015}. Investment in employee capacity development through training programs 
and certification initiatives has been shown to positively impact company performance in the global IT 
services industry \parencite{Chatterjee2017}. Furthermore, \textcite{Nowak2000} emphasize that intellectual 
capital constitutes a core value of the software industry. This highlights why the software services sector 
has drawn significant attention in research related to CSR reporting \parencite{Holder-Webb2009}.

The software industry has received particular attention in prior research examining the relationship 
between CSR reporting and financial performance \parencite{Okafor2021}.
However, social aspects, including employee relations and human rights ratings are not significantly associated 
with financial performance \parencite{Okafor2021}.

To date, there remains a lack of empirical research specifically addressing the social pillar of CSR reporting 
within the software industry. For example, \textcite{Holder-Webb2009} examined software companies within a sample 
of 50 firms across multiple industries in the United States. The results indicate that employment-related issues 
as well as health and safety are relatively well addressed in the software sector \parencite{Holder-Webb2009}. 
In contrast, other social aspects, such as human rights and community engagement, are not reported in a 
comprehensive manner \parencite{Holder-Webb2009}.

Moreover, the software industry experiences a rapid rate of technological change \parencite{Li2010}.
Such continuous changes require companies to rapidly adapt to new regulations, such as the ESRS. 

Reporting on key aspects such as work environment, work-life balance, 
and training is often inconsistent and rarely comprehensive \parencite{Greig2021}. Similarly,
disclosures on gender diversity were limited and linked to worse gender pay gaps,
suggesting a reputational motive \parencite{Huang2022}.

\textcite{Bornar2025} highlights that certain variables, including Average Training, Employee Accidents, 
Total Accidents, Employee Turnover, Employees with Disabilities, and Employee Injury Rate, exhibit 
a significant relationship with firms' financial performance indicators. In contrast, factors such as Gender Pay Gap, 
Trade Union Representation, and Employee Fatalities do not appear to be financially material. Additionally, 
several indicators, namely Employee Satisfaction, Women Managers, New Women Employees, and Lobbying Contribution, 
seem to have potential financial relevance but are not yet mandated in reporting requirements.
It has been found that mandatory disclosure requirements enhance social performance and that social 
and financial performance are positively associated.

In recent years, the importance of the social aspect within ESG has grown, according to \textcite{BaidJayaraman2022}. 

Additionally, contemporary political events have reinforced the relevance of the social component in ESG frameworks 
\parencite{She2022}.

Prior studies have highlighted 
persistent weaknesses in CSR report quality \parencite{DiChiacchio2024}, largely due to the scarcity of 
quantitative and monetary data \parencite{Michelon2015}.

\section{Prior Research on Drivers and Barriers of SSR Readiness}

\subsection{Drivers of SSR Readiness}









