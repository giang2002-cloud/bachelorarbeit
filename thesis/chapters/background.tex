\chapter{Literature Review and Theoretical Framework}
\label{chap:background}
\section{Literature Review}
\subsection{Prior Literature on Social Sustainability Reporting}

Sustainability reporting can be understood under various definitions, 
including corporate social responsibility (CSR), extended external reporting, 
and ESG reporting \parencite{Edge2022}. As defined by \parencite{Rasche2017}, CSR is the integration of social, 
environmental, ethical, and philanthropic responsibilities into business practice. 
Similarly, the ESG framework focuses on environmental, social, and governance. 
Across these approaches, the social dimension consistently emerges as a core aspect,
covering issues such as employment, working conditions and human rights \parencite{Fiechter2022, Morais2018}.

Several regulations and standards have been developed to provide a basis for social sustainability reporting.
Among these, the Global Reporting Initiative (GRI) is regarded as the most widely used framework 
by companies worldwide \parencite{Bais2024, vanOorschot2024}. 
According to GRI 401-406, companies disclose a wide range of social aspects, such as employment,
workplace conditions, equality issues and non-discrimination.
Marking a significant step forward, the European Sustainability Reporting Standards (ESRS) 
broaden this scope and make disclosure mandatory. The European Commission adopted the ESRS 
in July 2023 under the Corporate Sustainability Reporting Directive (CSRD). These standards 
oblige companies to report on impacts concerning the workforce (S1), value chain workers (S2), 
affected communities (S3), and consumers and end-users (S4).

Despite these regulatory developments, prior studies show that the social pillar remains underdeveloped 
within sustainability reporting practices. According to \textcite{Heldal2024}, the social dimension 
has not received adequate attention. This can be explained by \textcite{Morais2018}, 
who argue that the social pillar remains vague, difficult to quantify, and less prioritized than 
environmental issues. This view is reinforced by \textcite{Berg2022}, who demonstrate lower inter-rater 
consensus on social indicators compared to environmental ones.

\subsection{fghgf}

fff

\subsection{Prior Literature on SSR Readiness}

Most prior research examines the extent to which firms comply with the GRI Standards 
when issuing sustainability reports \parencite{Fonseca2014, Vigneau2015}. 
This emphasis is to be expected, as GRI is the most widely adopted global framework 
\parencite{Bebbington2012, Mahoney2013} and served as a key foundation 
for the development of the ESRS. Building on the GRI Standards, EFRAG developed 
the ESRS with a high degree of interoperability 
between the two frameworks, thereby facilitating companies' adaptation to the new mandatory 
reporting requirements. Given this close interrelation, the present study begins 
by reviewing prior literature on GRI-based compliance in sustainability (CSR) reporting, 
which provide a basis for the analysis of SSR readiness.

A study by \textcite{Mariappanadar2022} shows that companies often disclose only generic 
and anecdotal information concerning occupational health and safety (OHS) and employee welfare. 
Consequently, the level of compliance with GRI standards in these aspects remains low 
\parencite{Mariappanadar2022}. Although the volume of disclosures has increased, the primary 
motivation frequently lies in enhancing legitimacy and improving corporate image 
\parencite{Chauvey2015}. Multinational corporations (TNCs) often claim higher levels of 
compliance with GRI guidelines across labor and human rights indicators 
than what is actually practiced \parencite{Parsa2018}. According to \textcite{Parsa2018}, 
companies tend to avoid reporting on sensitive issues or those requiring detailed data, 
such as gender pay ratios or employee turnover disaggregated by age, gender, and region. 
Only a limited number of firms translate their statements on OHS and employee welfare 
into measurable sustainability outcomes \parencite{Mariappanadar2022}.


Most prior studies have addressed overall sustainability or CSR reporting readiness,
but have not examined social pillar readiness in isolation. 
At present, the introduction of the ESRS shifts attention toward firms' readiness to 
meet more comprehensive and mandatory reporting requirements. However, existing studies 
remain predominantly normative and conceptual, with limited empirical evidence 
on whether, and to what extent, firms have begun to adapt to the new sustainability 
reporting requirements. Notable exceptions are \textcite{Nicolo2025}, who examined early compliance with 
ESRS disclosure requirements among Italian listed firms and identified key firm-level drivers of 
adaptation, and \textcite{Filho2025}, who assessed the readiness of 20 firms drawn from a cross-sectoral 
sample of EU companies. \textcite{Nicolo2025} reports relatively high levels of early compliance 
with ESRS requirements among firms falling within the scope of the Omnibus Package. Whereas, 
\textcite{Raimo2025} documents relatively low overall levels of pre-compliance, 
particularly in integrated reports. \textcite{Filho2025} and \textcite{Montero2025} likewise reveal 
that disclosure, especially among SMEs, remains limited, due to the absence of concrete metrics, 
verification mechanisms, and external assurance. Overall, prior studies have focused on 
sustainability reporting in general and highlighted industry environmental sensitivity 
as a key determinant of readiness \parencite{Filho2025, Raimo2025, Nicolo2025, Montero2025}.



Overall, prior studies have predominantly 
concentrated on environmentally sensitive industries such as oil and energy \parencite{Filho2025, Raimo2025, Nicolo2025}, 
whereas the software sector has only been addressed within broader technology classifications \parencite{Montero2025}.


Reporting on key aspects such as work environment, work-life balance, 
and training is often inconsistent and rarely comprehensive \parencite{Greig2021}. Similarly,
disclosures on gender diversity were limited and linked to worse gender pay gaps,
suggesting a reputational motive \parencite{Huang2022}.


Prior studies have highlighted 
persistent weaknesses in CSR report quality \parencite{DiChiacchio2024}, largely due to the scarcity of 
quantitative and monetary data \parencite{Michelon2015}.


\section{Theoretical frameworks}

This study employs theoretical frameworks to explain the current level 
of social sustainability reporting readiness in the software services sector 
and to interpret the drivers and barriers shaping firms' preparedness.
Prior studies have highlighted the importance of theoretical perspectives 
in the study of sustainability reporting \parencite{Gesso2023, Rezaee2016, Lozano2015}.
Within the wide range of theories applied in this field, Institutional Theory
and Legitimacy Theory are among those frequently employed.

\subsection{Institutional Theory}
Institutional Theory is widely applied in sustainability 
reporting research \parencite{Campbell2007}. This theory explains 
how organizations adapt to social, political, and economic environments 
through institutionalized practices 
\parencite{Meyer1977, DiMaggio1983}.
It emphasizes how coercive, normative, and mimetic pressures shape organizational behavior.


